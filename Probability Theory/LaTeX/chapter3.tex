
\chapter{Measurable Functions and Integrating Functions}

We will now devote ourselves to the study of functions acting on measure spaces. 
First, we define what it means for a function to be measurable. Then, we build up 
our study of how to integrate functions. Throughout the chapter, we let 
$(\Omega, \F, \mu)$ be a fixed measure space.

\section{Measurable Functions}

\subsection{Intuition}

Let us return to chapter 1. There, we said that $\Omega$ could be an incredibly rich 
universe of possible events, but $\F$ is a subset of interest. For example, in a dice roll,
$\Omega$ could differentiate the outcomes of the dice roll, the weather tomorrow, and 
what you eat for lunch tomorrow. But we may let $\F = \sigma(\cup_{i=1}^6\{ \omega : \text{dice roll at $\omega$ is $i$}\})$ 
be the $\sigma$-algebra which captures enough richness for our purposes. We think of measurable 
functions as those which compose well with the set of events we care about. This is a sort of necessary 
but sufficient condition for our study of probability.

\begin{example}
    Let $\Omega = \{1,2,3,4,5,6\}, \mathcal F = \sigma(\{\{1,2\},\{3,4\}, \{5,6\}\})$ and 
    $\mu(\{1,2\}) = \mu(\{3,4\}) = \mu(\{5,6\}) = 1/3$. Finally, let $X(\omega) = 2\omega$.
\end{example}

In this example, $X$ is not measurable in some sense. Think about it. What is the probability that 
$X = 2$? Our measure is underspecified! Just because we know $\mu(X \in \{2,4\}) = 1/3$ doesn't mean we can 
determine $\mu(\{X \in 2\}) = \mu(\{1\})$. It could be that we have a biased die in which
$\mu(\{1\}) = 3/24, \mu(\{2\}) = 1/24$. So what is the expected value of $X$? There really isn't 
enough information! On the other hand, if $X = 1$ if $\omega \in \{1,2,3,4\}$ and is $0$ otherwise, 
we can determine the expected value of $X$, because $X^{-1}(\{1\}) = \{1,2,3,4\} \in \F$; likewise, 
$X^{-1}(\{0\}) = \{5,6\} \in \F$. 

\subsection{Definition of Measurability}

Suppose $(\Omega, \F, \mu)$ is a measure space and 
    $(\mathcal X, \mathcal A)$ is a family of sets equipped with 
    a $\sigma$-algebra.

\begin{definition}\label{def:measurable}
    A function $X : \Omega \to \mathcal X$ is said to be 
    $\F / \mathcal A$ measurable, or simply measurable, if for all 
    $A \in \mathcal A$, $X^{-1}(A) \in \mathcal F$. One may write 
    $X : (\Omega, \F) \to (\mathcal X, \mathcal A)$.
\end{definition}

\begin{definition}
    If $\mathcal X = \mathbb R, \mathcal A = \Borel$, the set of all $\F / \Borel$ measurable functions is 
    denoted $\mathcal M(\Omega, \Borel)$. The set of all such nonnegative functions is 
    $\mathcal M^+(\Omega, \Borel)$.
\end{definition}

\subsection{Determining Measurability}

Certainly, we could check ~\ref{def:measurable} by simply taking 
arbitrary elements of $\mathcal A$ and checking $X^{-1}(\mathcal A) \in \F$. 
This may prove to be a difficult task, however. For example, recall from our previous study
of the Lebesgue measure that $\Borel$ is complicated! We will show that 
it suffices to check a generating class.

\begin{theorem}\label{thm:measurable_sufficiency}
    If $\mathcal A = \sigma(\E)$ and for all $E \in \E$, 
    $X^{-1}(E) \in \F$, then $X$ is $\F / \mathcal A$ measurable.
\end{theorem}

\begin{proof}
    We proceed via a generating class argument. Let, 

    \[ \mathcal D = \{ A \in \mathcal A : X^{-1}(A) \in \F \} \]

    We shall show that $\mathcal D$ is a $\sigma$ algebra. First, observe that 
    $X^{-1}(\emptyset) = \emptyset \in \mathcal D$. Furthermore, if $A \in \mathcal D$, as 
    $X^{-1}(A^c) = X^{-1}(A)^c$, $X^{-1}(A) \in \F$, and $\F$ is closed under complement, 
    $A^c \in \mathcal D$. Thus $\mathcal D$ is closed under complement. Finally, suppose
    $A_1, A_2... \in \mathcal D$. One can check that $X^{-1}\big( \cup_{i=1}^\infty A_i \big) = \cup_{i=1}^\infty X^{-1}(A_i)$.
    And as each $X^{-1}(A_i) \in \F$, the whole countable union is as well. Thus, $\cup_{i=1}^\infty 
    A_i \in \mathcal D$. We conclude that $\mathcal D$ has the desirable closure properties of a 
    $\sigma$ algebra. \\ 

    By assumption, $\mathcal E \subseteq \mathcal D$. Therefore, 
    $\mathcal A = \sigma(\mathcal E) \subseteq \sigma(\mathcal D) = \mathcal D$. 
    Also, $\mathcal D \subseteq \mathcal A$ by definition, so $\mathcal A = \mathcal D$. 
    We conclude that $X$ is measurable.
\end{proof}

\subsection{Examples}

If $\F = \mathcal B(\mathbb R^{n}), \mathcal A = \Boreld$ and $X$ is continuous, 
then $X$ is measurable. It is a standard fact that continuous functions map open sets 
to open sets, and the preimage of an open set is open. Thus, let $\mathcal G_n$ be the open sets of $\R^n$ and 
$\mathcal G_d$ be the open sets of $\R^d$. Clearly, by this fact, for each 
$E \in \mathcal G_d, X^{-1}(E) \in \mathcal G_n$. And since $\mathcal G_d$ generates 
$\Boreld$, by theorem ~\ref{thm:measurable_sufficiency}, $X$ is measurable.

\subsection{Properties}

\begin{theorem}
If $X : (\Omega, \F) \to (\mathcal X, \mathcal A)$ and $Y : (\mathcal X, \mathcal A)\to (\mathcal Y, \mathcal B)$, then 
$Y \circ X : \Omega \to \mathcal Y$ is $\F / \mathcal B$ measurable.
\end{theorem}

\begin{proof}
    This is perfectly straightforward to check using the definitions.
\end{proof}

\begin{theorem}
    If $X, Y  \in \mathcal M(\Omega, \F)$ are both bounded and measurable, then 
    so is $X + Y$ and $XY$
\end{theorem}

\begin{proof}
    First, define $T = X + Y$. First, observe for any interval $(a,b)$,

    \[ T^{-1}(a,b) = \{\omega : X(\omega) + Y(\omega) > a\} \cap \{\omega : X(\omega) + Y(\omega) < b\}  \]

    Note that, 

    \[ \{\omega : X(\omega) + Y(\omega) > a\} = \bigcup_{q \in \mathbb Q}\{\omega : X(\omega) > q\} \cap \{\omega : Y(\omega) > a - q\}\]
    \[ = \bigcup_{q \in \mathbb Q} X^{-1}(q, \infty) \cap Y^{-1}(a-q,\infty)\]

    Note that $X^{-1}(q, \infty) \in \F, Y^{-1}(a-q,\infty) \in F$ by measurability. 
    And by closure properties of $\sigma$ algebras, the above is in $\F$. Likewise, 
    $  \{\omega : X(\omega) + Y(\omega) < b\} \in \F$. Again, by closure under intersection, 
    $T^{-1}(a,b) \in \F$. Since the open intervals generate $\Borel$, this is sufficient for measurability 
    by theorem ~\ref{thm:measurable_sufficiency}.\\ 

    Now, we show $XY$ is measurable in a somewhat similar fashion. We proceed like so: 
    \begin{itemize}
        \item Consider the map $T(\omega) = (X(\omega), Y(\omega))$ and 
            $\psi(u,v) = uv$. Note $XY = \psi \circ T$.
        \item As $\{ A \times B : A \in \Borel, B \in \Borel \}$ generates 
                $\mathcal B(\R^2)$, and $X$ and $Y$ are measurable, $T$ is measurable. 
                Why? Because, for $A,B \in \Borel$,

                \[ T^{-1}(A \times B) = \underbrace{X^{-1}(A)}_{\in \F} \cap \underbrace{Y^{-1}(B)}_{\in \F} \in \F  \]

                So by theorem ~\ref{thm:measurable_sufficiency}, this is sufficient to say that $T$ is measurable.

        \item As $\psi : \R^2 \to \R$ is continuous, it is measurable
        \item Thus, $XY$ can be regarded as the composition of measurable functions, so it is measurable.
    \end{itemize}
\end{proof}

\begin{theorem}
    If $X_1,X_2..$ is a sequence of measurable functions, then 
    $X = \sup_i X_i$ is measurable.
\end{theorem}

\begin{proof}
    Note intervals of the form $(a,\infty)$ generate $\Borel$. Furthermore, 

    \[ X^{-1}(a,\infty) = \{ \omega : X(\omega) > a \}  = \bigcup_{i=1}^n X_i^{-1}(a,\infty) \]

    Thus, $X^{-1}(a,\infty) \in \F$. Since these generate $\Borel$, we are done.
\end{proof}

\begin{theorem}
    If $X_1,X_2..$ is a sequence of measurable functions, then 
    $X = \inf_i X_i$ is measurable.
\end{theorem}

\begin{proof}
    The proof is analogous.
\end{proof}

\begin{Proposition}
    The $\lim\sup$ and $\lim\inf$ of measurable functions is measurable. 
\end{Proposition}

\begin{proof}
    We will show the proof for the $\lim\sup$ case as the $\lim\inf$ case is analogous. 
    Recall if $X_1,X_2...$ are measurable functions, then if $X = \lim\sup_i X_i$,

    \[ X(\omega) = \inf_n \sup_{m \geq n} X_i(\omega) \]

    As $\sup_{m \geq n} X_i(\omega)$ is measurable for each $n$, and the infimum of 
    measurable functions is measurable, $X$ is measurable.
\end{proof}

\section{The Integral}

We will now develop the notion of integrals of functions, from the ground up. 
First, we begin with a measure space $(\Omega, \F, \mu)$. Think of integrals as functionals: 
maps from the space of measurable functions to $\R$. While notation varies, we will 
adopt two ways of denoting the integral of a function $X$ with respect to a measure 
$\mu$:

\[ \int X d\mu \hspace{1cm} \text{and} \hspace{1cm} \mu(X) \]

While the $d\mu$ does relate to the $dx$ from Riemannian integration, ignore this for now. 
Think of the $d\mu$ merely as a symbol which says we are integrating with respect to 
$\mu$, rather than some other measure. 

\subsection{Simple Functions}

A simple function will be of the form, 

\[ X(\omega) = \sum_{i=1}^n \alpha_i \Ind\{A_i\} \]

Where each $\alpha_i \geq 0$ and $A_i \in \F$. For such an $X$, we will 
define its integral like so:

\[ \int X d\mu = \sum_{i=1}^n \alpha_i \mu(A_i) \]

If $\mu(A_i) = \infty, \alpha_i = 0$, adopt the convention that 
$\alpha_i \mu(A_i) = 0$ — this is the natural thing to do, as we don't want 
our integral to depend on sets which don't contribute to our function. It remains to verify consistency.

\begin{Proposition} 
    Suppose that $X$ can be written as $X = \sum_{j=1}^m \beta_j \Ind\{B_j\} =  \sum_{i=1}^n \alpha_i \Ind\{A_i\}$. 
Then, $\sum_{i=1}^n \alpha_i \mu(A_i) = \sum_{j=1}^m \beta_j \mu(B_j)$.
And so, the integral of a simple function is a well-defined object.
\end{Proposition}

\begin{proof}

Assume without loss of generality that $\cup_{j=1}^m B_j = \cup_{i=1}^n = A_i$. 
Otherwise, we could simply consider $\sum_{i=1}^n \alpha_i \Ind\{A_i\} + 0 \cdot \Ind\{\Omega - \cup_{i=1}^n A_i \}$ 
without changing $X$ or its integral. Furthermore, assume without loss of generality that the 
$A_i$'s are disjoint, as are the $B_j$'s. Let $\gamma_{i,j} = X(\omega)$ for $\omega \in A_i \cap B_j$.First, note that $X$ can be written as: 

\[ X = \sum_{i=1}^n \sum_{j=1}^m \alpha_i \Ind\{A_i \cap B_j\} =  \sum_{j=1}^m  \sum_{i=1}^n \beta_j \Ind\{B_j \cap A_i \} = \sum_{i=1}^n \sum_{j=1}^m \gamma_{i,j} \Ind\{A_i \cap B_j\}\]

Note for fixed $i, \gamma_{i,j} = \alpha_i$. For fixed $j$, $\gamma_{i,j} = \beta_j$. And so,

\[ \sum_{i=1}^n \alpha_i \mu(A_i) = \sum_{i=1}^n \sum_{j=1}^m \alpha_i \mu(A_i \cap B_j) \]
\[ = \sum_{i=1}^n \sum_{j=1}^m \gamma_{i,j} \mu(A_i \cap B_j) =  \sum_{j=1}^m \sum_{i=1}^n\gamma_{i,j} \mu(A_i \cap B_j) \]
\[ = \sum_{j=1}^m \sum_{i=1}^n\beta_j \mu(A_i \cap B_j) = \sum_{j=1}^m \beta_j \mu(B_j) \]

\end{proof}

\subsection{Properties of Integrals of Simple Functions}

\begin{Proposition} 
    $\int \alpha X + \beta Y d\mu = \alpha \int X d\mu + \beta \int Y d\mu$. Thus, the 
    integral is a linear functional.
\end{Proposition}

\begin{proof}
    Let, 

    \[ X = \sum_{i=1}^n \alpha_i \Ind\{A_i\}, Y = \sum_{j=1}^m \beta_j \Ind\{B_j\} \]

    Again, assume without loss of generality that the $A_i$'s are pairwise disjoint and 
    span $\Omega$. Then, 

    \[ \alpha X + \beta Y = \sum_{i=1}^n\sum_{j=1}^m (\alpha \alpha_i + \beta \beta_j)\Ind\{A_i \cap B_j\} \]

    Thus,

    \[ \int \alpha X + \beta Y d\mu  = \sum_{i=1}^n \sum_{j=1}^m (\alpha \alpha_i + \beta \beta_j)\mu(A_i \cap B_j)\]
    \[ =  \sum_{i=1}^n \sum_{j=1}^m\alpha \alpha_i \mu(A_i \cap B_j) + \sum_{j=1}^m \sum_{i=1}^n \beta\beta_j \mu(A_i \cap B_j)  \]
    \[ = \alpha \sum_{i=1}^n \alpha_i \mu(A_i) + \beta \sum_{j=1}^m \beta_j \mu(B_j) = \alpha \int X d\mu + \beta \int Y d\mu \] 
\end{proof}

    \subsection{Extension to All Measurable Functions}

    For a general measurable function $X \in \Mfp$, we define its integral to be the highest among those 
    simple functions which are less than $X$:

    \[ \int X d\mu = \sup_{s_n \leq X \text{ simple}}  \int s_n d\mu \]

    \begin{theorem}[This Integral is Nice]\label{thm:integral_properties}
        Suppose $X$ and $Y$ are measurable functions. Then, 
        \begin{enumerate}
            \item $\int \alpha X + \beta Y d\mu = \alpha \int X d\mu + \beta \int Y d\mu$
            \item If $X = Y$ almost everywhere, then $\int X d\mu = \int Y d\mu$.
            \item If $X \leq Y$ almost everywhere, then $\int X d\mu \leq \int Y d\mu$
        \end{enumerate}
    \end{theorem}
        \begin{proof}
            \textbf{Proof of 1}:  Let $x_n, y_n$ be simple functions 
            such that $\int X d\mu < \int x_n d\mu + 1/(2\alpha n)$, $\int Yd\mu < \int y_n d\mu + 1/(2\beta n)$. 
            Then, note $\alpha X + \beta Y \geq \alpha x_n + \beta y_n$. So, 

            \[ \int \alpha X + \beta Y d\mu \leq \int \alpha x_n + \beta y_nd\mu = \alpha \int x_n d\mu + \beta \int y_n d\mu \]
            \[ \leq \alpha \bigg(\int Xd\mu + 1/(2\alpha n)\bigg) +  \beta \bigg(\int Y d\mu + 1/(2\beta n)\bigg)  \]
            \[ = \alpha \int Xd\mu + \beta \int Y d\mu + 1/n \]

            Taking $n \to \infty$, we have $\int \alpha X + \beta Y d\mu \leq\alpha \int Xd\mu + \beta \int Y d\mu$. For 
            the reverse inequality, 

            \[  \int \alpha X + \beta Y d\mu = \sup_n \bigg\{ \int z_n d\mu : z_n \leq \alpha X + \beta Y \bigg\} \]
            \[ \geq \sup_n \bigg\{ \int \alpha x_n + \beta y_n d\mu : x_n \leq X, y_n \leq Y \bigg\}\]
            Since $x_n, y_n$ can vary freely,
            \[ =  \alpha \sup_n \bigg\{ \int x_n  d\mu : x_n \leq X \bigg\} + \beta \sup_n \bigg\{ \int  y_n  d\mu : y_n \leq Y \bigg\}  \]
            \[ = \alpha \int X d\mu + \beta \int Y d\mu \]
            Which gives the other side of the inequality. So we are done.\\

            \textbf{Proof of 2:} Let $N = \{\omega : X(\omega) \neq Y(\omega)\}$. By assumption, 
            $\mu(N) = 0$, so $N$ is negligible. Let $\tilde{X} = X\Ind\{N^c\}$. We show that 
            $\int X d\mu = \int \tilde{X} d\mu$. It then follows from symmetry and the fact that 
            $\tilde{Y} = \tilde{X}$ that the desired result is true. Note for any simple function $x = \sum_{i=1}^n \alpha_i \Ind\{A_i\}$, 

            \[ \int x d\mu = \sum_{i=1}^n \alpha_i \mu(A_i) =  \sum_{i=1}^n \alpha_i \mu(A_i \cap N^c) = \int x\Ind\{N^c\} d\mu  \]

            Thus, taking supremums, 

            \[ \int Xd\mu = \sup\bigg\{\int xd\mu : x \leq X \bigg\} = \sup\bigg\{\int x \Ind\{N^c\}d\mu : x \leq X \bigg\} \] 
            \[ = \sup\bigg\{\int x d\mu : x \leq X\Ind\{N^c\} \bigg\} = \int \tilde{Xd\mu} \]

            And thus, 

            \[ \int X d\mu = \int \tilde{X}d\mu = \int \tilde{Y}d\mu = \int Y d\mu \]

            \textbf{Proof of 3:} First, suppose $X \leq Y$ everywhere. The property will hold by definition of supremum. Note since $X \leq Y$, 

            \[ \int X d\mu = \sup\bigg\{ \int x : x \leq X\bigg\} \leq \sup\bigg\{ \int x : x \leq Y \bigg\} = \int Y d\mu \]

            Now, if $X > Y$ on a negligible set $N$. Consider, $\tilde{X} = X\Ind\{N^c\}$. Then 
            $\int \tilde{X}d\mu = \int X d\mu$ by property 2. And $\int \tilde{X}d\mu \leq \int Y d\mu$ by our earlier work, 
            so we are done!
        \end{proof}


    \begin{definition}[Convergence Definitions]
        We use the following conventions from here on: 
        \begin{itemize}
            \item Numbers: Say $a_1,a_2... \uparrow a$ if $a_1 \leq a_2...$ and $\lim_{n\to \infty}a_n = a$
            \item Sets: Say $A_1, A_2 \uparrow A$ if $A_1 \subseteq A_2 \subseteq A_3$... and $\cup_{i=1}^\infty A_i  = A$
            \item Sets: Say $A_1, A_2 \downarrow A$ if $A_1 \supseteq A_2 \supseteq A_3$... and $\cap_{i=1}^\infty A_i  = A$
            \item Functions: Say $X_n \uparrow X$ if $X_n \leq X_{n+1}$ for all $n$ and $X_n \to X$ pointwise
        \end{itemize}
    \end{definition}

    \begin{theorem}[The Monotone Convergence Theorem]
        If $X_n \uparrow X$ pointwise almost everywhere, then $X$ is measurable and $\int X_n d\mu \uparrow \int X d\mu$
    \end{theorem}

    The monotone convergence theorem is a fundamental result in the theory of integration. 
    To prove it, we will use the so called continuity of a measure: 

    \begin{theorem}
        If $A_n \downarrow \emptyset$, then $\mu(A_n) \downarrow 0$
    \end{theorem}

    \begin{proof}

    \end{proof}

    \begin{proof}
        $X$ is the supremum of measurable functions, so it is measurable. Note that it suffices to prove this theorem if $X_n \uparrow X$ everywhere by Theorem ~\ref{thm:integral_properties}. 
        First, for any finite $n$, $X_n \leq X$, so $\int X_n d\mu \leq \int X d\mu$. Taking the limit, we find, 
        
        \[ \lim_{n \to \infty}\int X_n d\mu \leq \int X d\mu \] 

        So it remains to show the opposite inequality. Let $B_{n,m} = \{\omega : X_n \geq X \frac{m-1}{m}\}$. Note that, 

        \[ \int X_n d\mu  = \int X_n B_{n,m} + \int X_n B_{n,m}^c \geq \frac{m-1}{m} \int X d\mu + \int X_n B_{n,m}^c  \]

        \begin{Proposition} 
            For all $m$, $\int X_n B_{n,m}^c \to 0$ as $n \to \infty$
        \end{Proposition}

        \begin{proof}
           Obviously, this integral is decreasing in $n$. Thus, it simply remains to 
           show that it can't get "stuck" above some $\epsilon$. First, observe that, 

           \[ \mu(B_{n,m}^c) \to 0 \text{ as } n \to \infty \]

           Since $B_{n,m}^c \to \emptyset$. 
        \end{proof}

        Thus, with the above, take $n \to \infty$: 

        \[ \int X_n d\mu \geq \frac{m-1}{m} \int X d\mu + 0 \]

        Now take $m \to \infty$: 

        \[ \int X_n d\mu \geq \int X d\mu \]

        And we are done.

    \end{proof}
