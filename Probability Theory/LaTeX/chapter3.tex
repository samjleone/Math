
\chapter{Measurable Functions and Integrating Functions}

We will now devote ourselves to the study of functions acting on measure spaces. 
First, we define what it means for a function to be measurable. Then, we build up 
our study of how to integrate functions. Throughout the chapter, we let 
$(\Omega, \F, \mu)$ be a fixed measure space.

\section{Measurable Functions}

\subsection{Intuition}

Let us return to chapter 1. There, we said that $\Omega$ could be an incredibly rich 
universe of possible events, but $\F$ is a subset of interest. For example, in a dice roll,
$\Omega$ could differentiate the outcomes of the dice roll, the weather tomorrow, and 
what you eat for lunch tomorrow. But we may let $\F = \sigma(\cup_{i=1}^6\{ \omega : \text{dice roll at $\omega$ is $i$}\})$ 
be the $\sigma$-algebra which captures enough richness for our purposes. We think of measurable 
functions as those which compose well with the set of events we care about. This is a sort of necessary 
but sufficient condition for our study of probability.

\begin{example}
    Let $\Omega = \{1,2,3,4,5,6\}, \mathcal F = \sigma(\{\{1,2\},\{3,4\}, \{5,6\}\})$ and 
    $\mu(\{1,2\}) = \mu(\{3,4\}) = \mu(\{5,6\}) = 1/3$. Finally, let $X(\omega) = 2\omega$.
\end{example}

In this example, $X$ is not measurable in some sense. Think about it. What is the probability that 
$X = 2$? Our measure is underspecified! Just because we know $\mu(X \in \{2,4\}) = 1/3$ doesn't mean we can 
determine $\mu(\{X \in 2\}) = \mu(\{1\})$. It could be that we have a biased die in which
$\mu(\{1\}) = 3/24, \mu(\{2\}) = 1/24$. So what is the expected value of $X$? There really isn't 
enough information! On the other hand, if $X = 1$ if $\omega \in \{1,2,3,4\}$ and is $0$ otherwise, 
we can determine the expected value of $X$, because $X^{-1}(\{1\}) = \{1,2,3,4\} \in \F$; likewise, 
$X^{-1}(\{0\}) = \{5,6\} \in \F$. 

\subsection{Definition of Measurability}

Suppose $(\Omega, \F, \mu)$ is a measure space and 
    $(\mathcal X, \mathcal A)$ is a family of sets equipped with 
    a $\sigma$-algebra.

\begin{definition}\label{def:measurable}
    A function $X : \Omega \to \mathcal X$ is said to be 
    $\F / \mathcal A$ measurable, or simply measurable, if for all 
    $A \in \mathcal A$, $X^{-1}(A) \in \mathcal F$. One may write 
    $X : (\Omega, \F) \to (\mathcal X, \mathcal A)$.
\end{definition}

\begin{definition}
    If $\mathcal X = \mathbb R, \mathcal A = \Borel$, the set of all $\F / \Borel$ measurable functions is 
    denoted $\mathcal M(\Omega, \Borel)$. The set of all such nonnegative functions is 
    $\mathcal M^+(\Omega, \Borel)$.
\end{definition}

\subsection{Determining Measurability}

Certainly, we could check ~\ref{def:measurable} by simply taking 
arbitrary elements of $\mathcal A$ and checking $X^{-1}(\mathcal A) \in \F$. 
This may prove to be a difficult task, however. For example, recall from our previous study
of the Lebesgue measure that $\Borel$ is complicated! We will show that 
it suffices to check a generating class.

\begin{theorem}\label{thm:measurable_sufficiency}
    If $\mathcal A = \sigma(\E)$ and for all $E \in \E$, 
    $X^{-1}(E) \in \F$, then $X$ is $\F / \mathcal A$ measurable.
\end{theorem}

\begin{proof}
    We proceed via a generating class argument. Let, 

    \[ \mathcal D = \{ A \in \mathcal A : X^{-1}(A) \in \F \} \]

    We shall show that $\mathcal D$ is a $\sigma$ algebra. First, observe that 
    $X^{-1}(\emptyset) = \emptyset \in \mathcal D$. Furthermore, if $A \in \mathcal D$, as 
    $X^{-1}(A^c) = X^{-1}(A)^c$, $X^{-1}(A) \in \F$, and $\F$ is closed under complement, 
    $A^c \in \mathcal D$. Thus $\mathcal D$ is closed under complement. Finally, suppose
    $A_1, A_2... \in \mathcal D$. One can check that $X^{-1}\big( \cup_{i=1}^\infty A_i \big) = \cup_{i=1}^\infty X^{-1}(A_i)$.
    And as each $X^{-1}(A_i) \in \F$, the whole countable union is as well. Thus, $\cup_{i=1}^\infty 
    A_i \in \mathcal D$. We conclude that $\mathcal D$ has the desirable closure properties of a 
    $\sigma$ algebra. \\ 

    By assumption, $\mathcal E \subseteq \mathcal D$. Therefore, 
    $\mathcal A = \sigma(\mathcal E) \subseteq \sigma(\mathcal D) = \mathcal D$. 
    Also, $\mathcal D \subseteq \mathcal A$ by definition, so $\mathcal A = \mathcal D$. 
    We conclude that $X$ is measurable.
\end{proof}

\subsection{Examples}

If $\F = \mathcal B(\mathbb R^{n}), \mathcal A = \Boreld$ and $X$ is continuous, 
then $X$ is measurable. It is a standard fact that continuous functions map open sets 
to open sets, and the preimage of an open set is open. Thus, let $\mathcal G_n$ be the open sets of $\R^n$ and 
$\mathcal G_d$ be the open sets of $\R^d$. Clearly, by this fact, for each 
$E \in \mathcal G_d, X^{-1}(E) \in \mathcal G_n$. And since $\mathcal G_d$ generates 
$\Boreld$, by theorem ~\ref{thm:measurable_sufficiency}, $X$ is measurable.

\subsection{Properties}

\begin{theorem}
If $X : (\Omega, \F) \to (\mathcal X, \mathcal A)$ and $Y : (\mathcal X, \mathcal A)\to (\mathcal Y, \mathcal B)$, then 
$Y \circ X : \Omega \to \mathcal Y$ is $\F / \mathcal B$ measurable.
\end{theorem}

\begin{proof}
    This is perfectly straightforward to check using the definitions.
\end{proof}

\begin{theorem}
    If $X, Y  \in \mathcal M(\Omega, \F)$ are both bounded and measurable, then 
    so is $X + Y$ and $XY$
\end{theorem}

\begin{proof}
    First, define $T = X + Y$. First, observe for any interval $(a,b)$,

    \[ T^{-1}(a,b) = \{\omega : X(\omega) + Y(\omega) > a\} \cap \{\omega : X(\omega) + Y(\omega) < b\}  \]

    Note that, 

    \[ \{\omega : X(\omega) + Y(\omega) > a\} = \bigcup_{q \in \mathbb Q}\{\omega : X(\omega) > q\} \cap \{\omega : Y(\omega) > a - q\}\]
    \[ = \bigcup_{q \in \mathbb Q} X^{-1}(q, \infty) \cap Y^{-1}(a-q,\infty)\]

    Note that $X^{-1}(q, \infty) \in \F, Y^{-1}(a-q,\infty) \in F$ by measurability. 
    And by closure properties of $\sigma$ algebras, the above is in $\F$. Likewise, 
    $  \{\omega : X(\omega) + Y(\omega) < b\} \in \F$. Again, by closure under intersection, 
    $T^{-1}(a,b) \in \F$. Since the open intervals generate $\Borel$, this is sufficient for measurability 
    by theorem ~\ref{thm:measurable_sufficiency}.\\ 

    Now, we show $XY$ is measurable in a somewhat similar fashion. We proceed like so: 
    \begin{itemize}
        \item Consider the map $T(\omega) = (X(\omega), Y(\omega))$ and 
            $\psi(u,v) = uv$. Note $XY = \psi \circ T$.
        \item As $\{ A \times B : A \in \Borel, B \in \Borel \}$ generates 
                $\mathcal B(\R^2)$, and $X$ and $Y$ are measurable, $T$ is measurable. 
                Why? Because, for $A,B \in \Borel$,

                \[ T^{-1}(A \times B) = \underbrace{X^{-1}(A)}_{\in \F} \cap \underbrace{Y^{-1}(B)}_{\in \F} \in \F  \]

                So by theorem ~\ref{thm:measurable_sufficiency}, this is sufficient to say that $T$ is measurable.

        \item As $\psi : \R^2 \to \R$ is continuous, it is measurable
        \item Thus, $XY$ can be regarded as the composition of measurable functions, so it is measurable.
    \end{itemize}
\end{proof}

\begin{theorem}
    If $X_1,X_2..$ is a sequence of measurable functions, then 
    $X = \sup_i X_i$ is measurable.
\end{theorem}

\begin{proof}
    Note intervals of the form $(a,\infty)$ generate $\Borel$. Furthermore, 

    \[ X^{-1}(a,\infty) = \{ \omega : X(\omega) > a \}  = \bigcup_{i=1}^n X_i^{-1}(a,\infty) \]

    Thus, $X^{-1}(a,\infty) \in \F$. Since these generate $\Borel$, we are done.
\end{proof}

\begin{theorem}
    If $X_1,X_2..$ is a sequence of measurable functions, then 
    $X = \inf_i X_i$ is measurable.
\end{theorem}

\begin{proof}
    The proof is analogous.
\end{proof}

\begin{Proposition}
    The $\lim\sup$ and $\lim\inf$ of measurable functions is measurable. 
\end{Proposition}

\begin{proof}
    We will show the proof for the $\lim\sup$ case as the $\lim\inf$ case is analogous. 
    Recall if $X_1,X_2...$ are measurable functions, then if $X = \lim\sup_i X_i$,

    \[ X(\omega) = \inf_n \sup_{m \geq n} X_i(\omega) \]

    As $\sup_{m \geq n} X_i(\omega)$ is measurable for each $n$, and the infimum of 
    measurable functions is measurable, $X$ is measurable.
\end{proof}

\section{The Integral}

We will now develop the notion of integrals of functions, from the ground up. 
First, we begin with a measure space $(\Omega, \F, \mu)$. Think of integrals as functionals: 
maps from the space of measurable functions to $\R$. While notation varies, we will 
adopt two ways of denoting the integral of a function $X$ with respect to a measure 
$\mu$:

\[ \int X d\mu \hspace{1cm} \text{and} \hspace{1cm} \mu(X) \]

While the $d\mu$ does relate to the $dx$ from Riemannian integration, ignore this for now. 
Think of the $d\mu$ merely as a symbol which says we are integrating with respect to 
$\mu$, rather than some other measure. 

\subsection{Simple Functions}

A simple function will be of the form, 

\[ X(\omega) = \sum_{i=1}^n \alpha_i \Ind\{A_i\} \]

Where each $\alpha_i \geq 0$ and $A_i \in \F$. For such an $X$, we will 
define its integral like so:

\[ \int X d\mu = \sum_{i=1}^n \alpha_i \mu(A_i) \]

If $\mu(A_i) = \infty, \alpha_i = 0$, adopt the convention that 
$\alpha_i \mu(A_i) = 0$ — this is the natural thing to do, as we don't want 
our integral to depend on sets which don't contribute to our function. It remains to verify consistency.

\begin{Proposition} 
    Suppose that $X$ can be written as $X = \sum_{j=1}^m \beta_j \Ind\{B_j\} =  \sum_{i=1}^n \alpha_i \Ind\{A_i\}$. 
Then, $\sum_{i=1}^n \alpha_i \mu(A_i) = \sum_{j=1}^m \beta_j \mu(B_j)$.
And so, the integral of a simple function is a well-defined object.
\end{Proposition}

\begin{proof}

Assume without loss of generality that $\cup_{j=1}^m B_j = \cup_{i=1}^n = A_i$. 
Otherwise, we could simply consider $\sum_{i=1}^n \alpha_i \Ind\{A_i\} + 0 \cdot \Ind\{\Omega - \cup_{i=1}^n A_i \}$ 
without changing $X$ or its integral. Furthermore, assume without loss of generality that the 
$A_i$'s are disjoint, as are the $B_j$'s. Let $\gamma_{i,j} = X(\omega)$ for $\omega \in A_i \cap B_j$.First, note that $X$ can be written as: 

\[ X = \sum_{i=1}^n \sum_{j=1}^m \alpha_i \Ind\{A_i \cap B_j\} =  \sum_{j=1}^m  \sum_{i=1}^n \beta_j \Ind\{B_j \cap A_i \} = \sum_{i=1}^n \sum_{j=1}^m \gamma_{i,j} \Ind\{A_i \cap B_j\}\]

Note for fixed $i, \gamma_{i,j} = \alpha_i$. For fixed $j$, $\gamma_{i,j} = \beta_j$. And so,

\[ \sum_{i=1}^n \alpha_i \mu(A_i) = \sum_{i=1}^n \sum_{j=1}^m \alpha_i \mu(A_i \cap B_j) \]
\[ = \sum_{i=1}^n \sum_{j=1}^m \gamma_{i,j} \mu(A_i \cap B_j) =  \sum_{j=1}^m \sum_{i=1}^n\gamma_{i,j} \mu(A_i \cap B_j) \]
\[ = \sum_{j=1}^m \sum_{i=1}^n\beta_j \mu(A_i \cap B_j) = \sum_{j=1}^m \beta_j \mu(B_j) \]

\end{proof}

\subsection{Properties of Integrals of Simple Functions}

\begin{Proposition} 
    $\int \alpha X + \beta Y d\mu = \alpha \int X d\mu + \beta \int Y d\mu$. Thus, the 
    integral is a linear functional.
\end{Proposition}

\begin{proof}
    Let, 

    \[ X = \sum_{i=1}^n \alpha_i \Ind\{A_i\}, Y = \sum_{j=1}^m \beta_j \Ind\{B_j\} \]

    Again, assume without loss of generality that the $A_i$'s are pairwise disjoint and 
    span $\Omega$. Then, 

    \[ \alpha X + \beta Y = \sum_{i=1}^n\sum_{j=1}^m (\alpha \alpha_i + \beta \beta_j)\Ind\{A_i \cap B_j\} \]

    Thus,

    \[ \int \alpha X + \beta Y d\mu  = \sum_{i=1}^n \sum_{j=1}^m (\alpha \alpha_i + \beta \beta_j)\mu(A_i \cap B_j)\]
    \[ =  \sum_{i=1}^n \sum_{j=1}^m\alpha \alpha_i \mu(A_i \cap B_j) + \sum_{j=1}^m \sum_{i=1}^n \beta\beta_j \mu(A_i \cap B_j)  \]
    \[ = \alpha \sum_{i=1}^n \alpha_i \mu(A_i) + \beta \sum_{j=1}^m \beta_j \mu(B_j) = \alpha \int X d\mu + \beta \int Y d\mu \] 
\end{proof}

    \subsection{Extension to All Measurable Functions}

    For a general measurable function $X \in \Mfp$, we define its integral to be the highest among those 
    simple functions which are less than $X$:

    \[ \int X d\mu = \sup_{s_n \leq X \text{ simple}}  \int s_n d\mu \]

    \begin{theorem}[This Integral is Nice]\label{thm:integral_properties}
        Suppose $X$ and $Y$ are measurable functions. Then, 
        \begin{enumerate}
            \item $\int \alpha X + \beta Y d\mu = \alpha \int X d\mu + \beta \int Y d\mu$
            \item If $X = Y$ almost everywhere, then $\int X d\mu = \int Y d\mu$.
            \item If $X \leq Y$ almost everywhere, then $\int X d\mu \leq \int Y d\mu$
        \end{enumerate}
    \end{theorem}
        \begin{proof}
            \textbf{Proof of 1}:  Let $x_n, y_n$ be simple functions 
            such that $\int X d\mu < \int x_n d\mu + 1/(2\alpha n)$, $\int Yd\mu < \int y_n d\mu + 1/(2\beta n)$. 
            Then, note $\alpha X + \beta Y \geq \alpha x_n + \beta y_n$. So, 

            \[ \int \alpha X + \beta Y d\mu \leq \int \alpha x_n + \beta y_nd\mu = \alpha \int x_n d\mu + \beta \int y_n d\mu \]
            \[ \leq \alpha \bigg(\int Xd\mu + 1/(2\alpha n)\bigg) +  \beta \bigg(\int Y d\mu + 1/(2\beta n)\bigg)  \]
            \[ = \alpha \int Xd\mu + \beta \int Y d\mu + 1/n \]

            Taking $n \to \infty$, we have $\int \alpha X + \beta Y d\mu \leq\alpha \int Xd\mu + \beta \int Y d\mu$. For 
            the reverse inequality, 

            \[  \int \alpha X + \beta Y d\mu = \sup_n \bigg\{ \int z_n d\mu : z_n \leq \alpha X + \beta Y \bigg\} \]
            \[ \geq \sup_n \bigg\{ \int \alpha x_n + \beta y_n d\mu : x_n \leq X, y_n \leq Y \bigg\}\]
            Since $x_n, y_n$ can vary freely,
            \[ =  \alpha \sup_n \bigg\{ \int x_n  d\mu : x_n \leq X \bigg\} + \beta \sup_n \bigg\{ \int  y_n  d\mu : y_n \leq Y \bigg\}  \]
            \[ = \alpha \int X d\mu + \beta \int Y d\mu \]
            Which gives the other side of the inequality. So we are done.\\

            \textbf{Proof of 2:} Let $N = \{\omega : X(\omega) \neq Y(\omega)\}$. By assumption, 
            $\mu(N) = 0$, so $N$ is negligible. Let $\tilde{X} = X\Ind\{N^c\}$. We show that 
            $\int X d\mu = \int \tilde{X} d\mu$. It then follows from symmetry and the fact that 
            $\tilde{Y} = \tilde{X}$ that the desired result is true. Note for any simple function $x = \sum_{i=1}^n \alpha_i \Ind\{A_i\}$, 

            \[ \int x d\mu = \sum_{i=1}^n \alpha_i \mu(A_i) =  \sum_{i=1}^n \alpha_i \mu(A_i \cap N^c) = \int x\Ind\{N^c\} d\mu  \]

            Thus, taking supremums, 

            \[ \int Xd\mu = \sup\bigg\{\int xd\mu : x \leq X \bigg\} = \sup\bigg\{\int x \Ind\{N^c\}d\mu : x \leq X \bigg\} \] 
            \[ = \sup\bigg\{\int x d\mu : x \leq X\Ind\{N^c\} \bigg\} = \int \tilde{Xd\mu} \]

            And thus, 

            \[ \int X d\mu = \int \tilde{X}d\mu = \int \tilde{Y}d\mu = \int Y d\mu \]

            \textbf{Proof of 3:} First, suppose $X \leq Y$ everywhere. The property will hold by definition of supremum. Note since $X \leq Y$, 

            \[ \int X d\mu = \sup\bigg\{ \int x : x \leq X\bigg\} \leq \sup\bigg\{ \int x : x \leq Y \bigg\} = \int Y d\mu \]

            Now, if $X > Y$ on a negligible set $N$. Consider, $\tilde{X} = X\Ind\{N^c\}$. Then 
            $\int \tilde{X}d\mu = \int X d\mu$ by property 2. And $\int \tilde{X}d\mu \leq \int Y d\mu$ by our earlier work, 
            so we are done!
        \end{proof}


    \begin{definition}[Convergence Definitions]
        We use the following conventions from here on: 
        \begin{itemize}
            \item Numbers: Say $a_1,a_2... \uparrow a$ if $a_1 \leq a_2...$ and $\lim_{n\to \infty}a_n = a$
            \item Sets: Say $A_1, A_2 \uparrow A$ if $A_1 \subseteq A_2 \subseteq A_3$... and $\cup_{i=1}^\infty A_i  = A$
            \item Sets: Say $A_1, A_2 \downarrow A$ if $A_1 \supseteq A_2 \supseteq A_3$... and $\cap_{i=1}^\infty A_i  = A$
            \item Functions: Say $X_n \uparrow X$ if $X_n \leq X_{n+1}$ for all $n$ and $X_n \to X$ pointwise
        \end{itemize}
    \end{definition}

    The monotone convergence theorem is a fundamental result in the theory of integration. 
    To prove it, we will use the so called continuity of a measure: 

    \begin{theorem}
        If $A_n \downarrow \emptyset$ with at least one $\mu(A_i) < \infty$, then $\mu(A_n) \downarrow 0$. 
        Also, if $A_n \uparrow A$, then $\mu(A_n) \uparrow \mu(A)$.
    \end{theorem}

    \begin{proof}
        Assume WLOG that $\mu(A_1) < \infty$. Let $B_1 = A_1 \setminus A_2, B_2 = A_2 \setminus A_3$, and so on: $B_i = A_{i-1} \setminus A_i$. 
        It's clear that $\cup_i B_i = A_1$. Thus, by countable additivity,
        
        \[ \mu(A_1) = \sum_{i=1}^\infty \mu(B_i) \]

        On the other hand, 

        \[ \mu(A_1) = \sum_{i=1}^n \mu(B_i) + \sum_{i=n+1}^\infty \mu(B_i) = \sum_{i=1}^n \mu(B_i) + \mu(A_n) \]

        Taking $n \to \infty$, it must be that $\sum_{i=1}^n \mu(B_i) \to \mu(A_1)$. And thus, for equality to hold, 
        $\mu(A_n) \downarrow 0$. The proof of the second statement works in much the same way.
    \end{proof}

    \begin{theorem}[The Monotone Convergence Theorem]
        If $X_n \uparrow X$ pointwise almost everywhere, then $X$ is measurable and $\int X_n d\mu \uparrow \int X d\mu$
    \end{theorem}

    \begin{proof}
        
        First, assume without loss of generality that $X_n \uparrow X$ everywhere, as theorem ~\ref{thm:integral_properties} would readily 
        imply the desired result. Also, the measurability of $X$ has already been proven, as it is the $\sup$ of measurable 
        functions. Furthemore, for any $n$, $\int X_n d\mu \leq \int X d\mu$ as $X_n \leq X$. So taking 
        $n \to \infty$, we find $\lim_n \int X_n d\mu \leq \int X d\mu$. It remains to show that 
        $\lim_n \int X_n d\mu \geq \int X d\mu$. Indeed, let $X_m$ be a sequence of simple 
        functions such that $\int X_m d\mu > \int X - 1/m$. Now define $X_{n,m} = X_m (1-\frac{1}{m})\Ind\{X_n \geq (1-\frac{1}{m}X_m)\}$.
        Note that $X_{n,m}$ is simple, and $X_{n,m} \leq X_n$. And thus, 
        \[ \int X_{n,m} d\mu \leq \int X_n d\mu \]
        Now assume that $X_m = \sum_{i=1}^{n_m} \alpha_{i,m} \Ind\{A_{i,m}\}$ so that $X_{n,m} = (1-\frac{1}{m}) \sum_{i=1}^{n_m} \alpha_{i,m} \Ind\{A_{i,m} \cap \{X_n \geq (1-\frac{1}{m}X_m) \} \}$.
        Thus, 
        \[ \int X_n d\mu \geq (1-\frac{1}{m}) \sum_{i=1}^{n_m}\alpha_{i,m} \mu(\{A_{i,m} \cap \{X_n \geq (1-\frac{1}{m}X_m) \} \}) \] 
        Taking $n \to \infty$, the continuity of measures implies that $A_{i,m} \cap \{X_n \geq (1-\frac{1}{m}X_m)\} \uparrow A_{i,m}$. Thus, 

        \[ \lim_{n \to \infty} \int X_n d\mu \geq (1-\frac{1}{m}) \int X_m d\mu \geq (1 - \frac{1}{m}) \bigg( \int X d\mu - \frac{1}{m} \bigg)\] 

        Taking $m \to \infty$, we are done!

    \end{proof}

    \subsection{Integrals of More General Functions}

    Our last iteration of building the integral is that for possibly negative functions. 
    For $X \in \Mf$, let $X^+ = X\Ind\{X \geq 0\}$ and $X^- = |X|\Ind\{X \leq 0\}$. It follows that 
    $X = X^+ - X^-$. If at least one of $\int X^+d\mu, \int X^- d\mu$ is finite, we define $\int X d\mu
     = \int X^+ d\mu - \int X^- d\mu$. Otherwise, the integral is not defined. If the quantity is finite, the 
     function is said to be integrable.
      One can easily verify the 
    same niceness properties as before by splitting arbitrary functions into their positive and negative parts. 

    \section{Limit Theorems}

    \subsection{Fatou's Lemma}

    Fatou's Lemma is a relatively simple theorem to prove which will allow us 
    to prove many limit results down the road. 

    \begin{lemma}[Fatou's Lemma]
        For $\{X_n\}_n \in \Mfp$, $\int \lim\inf_n X_n d\mu \leq \lim\inf \int X_n d\mu$. 
    \end{lemma}

    \begin{proof}
        First, to remember this inequality, think of the $\lim\inf$ of a function 
        as containing many more degrees of freedom than the $\lim\inf$ of the integral, 
        and thus we are lower. Note first that, for all $n$, $\inf_n X_n \leq X_n$. 
        Define $Y_m = \inf_{n \geq m} X_n$. The monotone convergence theorem implies, 

        \[ \lim_{m \to \infty} \int Y_m = \int \sup Y_m \]

        And thus, substituting our definitions, 

        \[  \lim_{m \to \infty}  \int \inf_{n \geq m} X_n d\mu = \int \sup_m \inf_{n \geq m} X_n  = \int \lim\inf_n X_n d\mu \]

        Note also that for all $m$, $Y_m \leq X_m$. Thus, 

        \[  \lim\inf_m \int Y_m d\mu \leq \lim\inf_m \int X_m d\mu \]

        But when a limit exists, it is equal to the $\lim\inf$, so,

        \[\int \lim\inf_n X_n d\mu  =  \lim\inf_m \int Y_m d\mu \leq \lim\inf_m \int X_m d\mu \]

        And we're done!

    \end{proof}


    \subsection{The Dominated Convergence Theorem}

    Next comes one of the most important results yet! It comes up time and time again and is 
    arguably the most general tool or interchanging limits. 

    \begin{theorem}[The Dominated Convergence Theorem]
        Let $X_n$ converge pointwise to a function $X$ almost everywhere. 
        Assume $|X_n| \leq Y$ for some integrable function $Y$. Then 
        $\lim_n \int X_n d\mu = \int X d\mu$.
    \end{theorem}

    \begin{proof}
        As $Y$ is integrable, each $X_n$ is integrable. Furthermore, 
        $Y - X_n, Y + X_n$ are integrable and nonnegative, by the tringle inequality. Therefore, 

        \[  \int \lim\inf_n(Y - X_n) d\mu \leq \lim\inf_n\int(Y - X_n) d\mu \]
        \[  \int \lim\inf_n(Y + X_n) d\mu \leq \lim\inf_n \int(Y + X_n) d\mu \]

        By linearity of the integral and the $\lim\inf$, this implies,

        \[ \int \lim\inf_n (-X_n) d\mu \leq  \lim\inf_n \bigg(-\int X_nd\mu\bigg) \]
        \[ \int \lim\inf_n X_n  d\mu \leq \lim\inf_n \int X_n d\mu \ \]

        Note the first line is equivalent to, 

        \[ \int \lim\sup_n X_n d\mu \geq \lim\sup_n \int X_n d\mu  \]

        But we assumed that $X_n$ converges, so $\lim\inf_n X_n = \lim\sup_n X_n = X$. Combining, 
        we obtain, 

        \[ \lim\sup_n \int_n X_n d\mu \leq \int X d\mu \leq \lim\inf_n \int X_n d\mu \]

        But obviously, $\lim\sup_n \int_n X_n \geq \lim\inf_n \int_n X_n$. This forces their 
        equality, and thus the assertion follows as claimed.

        \subsection{Application: Differentiation Under the Integral}

        Oftentimes, you will see something of the essence of: 

        \[ \frac{\partial}{\partial t} \int f(x,t)dx =\int \frac{\partial}{\partial t}  f(x,t)dx \]

        How could this be true? It could be easily understood as an application of the dominated convergence theorem. 
        Indeed, consider for some $n$, 

        \[ X_{n,t}(x) = n(f(x,t + 1/n) - f(x,t)) \]

        It is clear to see that, for fixed $x$, $\lim_{n \to \infty}X_n(x,t) = \frac{\partial}{\partial t}f(x,t)$. 
        Thus, the tempting interchange of limits is:

        \[ \int \frac{\partial}{\partial t}f(x,t) dx = \int \lim_n X_{n,t} dx = \lim_n \int X_{n,t} dx = \lim_n \int X_{n,t} dx \]
        \[ = \lim_n n \bigg(\int f(x,t+1/n)dx - \int f(x,t)dx\bigg) = \frac{\partial}{\partial t} \int f(x,t)dx \]

        How do we make this rigorous? Well first obesrve that, in a way, this is a claim about 
        differentiation at each $t$. So fix $t$. For dominated convergence, we require that
        $X_{n,t}$ be dominated by an integrable function $Y_t(x)$, where this function is allowed to depend on 
        $t$, (but not on $n$ — so there is some dependence on $t$). If $f$ is continuously differentiable with respect to 
        $t$, it suffices to bound $\frac{\partial}{\partial t}f(x,t)$ at $t$, since then this can be extended to a more 
        local bound. Of course, nothing is stopping us from collecting our bounding functions into 
        a function of two variables. Summarizing our analysis into a theorem, we have: 

        \begin{theorem}
            Let $f(x,t) : \R^2 \to \R$ be a function which is continuous differentiable in an open set with respect to 
            $t$. Suppose there exists a function $F(x,t)$ which is integrable for fixed $x$ and 
            $|\frac{\partial}{\partial t}f(x,t)| \leq F(x,t)$. Then, $\frac{\partial}{\partial t} \int f(x,t)dx =\int \frac{\partial}{\partial t}  f(x,t)dx$ 
            for all $t$.
        \end{theorem}

    \end{proof}

    \section{The Borel Cantelli Lemma}

    This is a bit of an aside about a probabilistic technique that will come up often. It works very simply. 

    \begin{definition} 
        As a matter of notation, for a countable family of sets $A_1,A_2...$, define, 

        \[ \{A_n \: i.o. \} = \bigcap_{n=1}^\infty \bigcup_{m=n}^\infty A_m = \{ \omega : \omega \text{ in infinitely many } A_n \} \]

        This set can be understood as those events which occur ``infinitely" often in the sequence $A_1,A_2..$
    \end{definition}
    \begin{lemma}[The Borel Cantelli Lemma (Part 1)]
        Let $(\Omega, \F, \mathbb P)$ be a probability space. Then if 
        $\sum_{n=1}^\infty \mathbb P(A_n) < \infty$, $\mathbb P(\{A_n \: i.o.\}) = 0$.
    \end{lemma}

    \begin{proof}
        Note if $\sum_n \mathbb P(A_n) < \infty$, then $\mathbb P(A_n) \to 0$. Additionally, 
        $\lim_n \sum_{m=n}^\infty \mathbb P(A_m) \to 0$. And thus, for any $n$,
        \[ \mathbb P\{A_n \: i.o.\} \leq \mathbb P\{\bigcup_{m=n}^\infty A_m\} \leq \sum_{m = n}^\infty \mathbb P(A_m) \]
        Taking $n \to \infty$, the right hand side becomes $0$. This proves part 1 of the lemma. We will 
        be able to prove a converse in a Part 2 once we introduce notions of independence.
    \end{proof}

    \section{Special Case: Expectations}

    When $\mu = \mathbb P$ is a probability measure, the corresponding integral corresponds to an expectation. 
    In this, we write, 

    \[ \int X d\Prob = \mathbb E[X] \]

    We will typically adopt the latter when the measure is unambiguous. 

    \section{Convexity}

    \subsection{Convex Combinations}

    Recall that a convex function $f : U \to V$, where $U,V \subseteq \mathbb R$, has the property that for all 
    $x,y \in U, \lambda \in [0,1]$,

    \[ f(\lambda x + (1-\lambda) y) \leq \lambda f(x) + (1-\lambda) f(y) \]

    Note that $\lambda$ and $1-\lambda$ provide a so called \emph{convex combination} of numbers. 
    In general, $a_1, a_2...a_n$ is called a convex combination if $a_i > 0$ for all $i$ and 
    $\sum_{i=1}^n a_i = 1$. One can easily verify that the above relation holds for more 
    general convex combinations: 

    \begin{lemma}
        Let $a_1..a_n$ be a convex combination with $x_1,x_2...x_n \in U$. Then, 
        $f(\sum_{i=1}^n a_i x_i) \leq \sum_{i=1}^n a_i f(x_i)$.
    \end{lemma}
    
    \begin{proof}
        We proceed by induction. The $n=2$ case is the direct definition of convexity. 
        Now suppose the $n-1$ case is true. First, suppose $0 < a_n < 1$; otherwise, the 
        inductive step is trivial. We have,

        \[ f(\sum_{i=1}^n a_i x_i) = f(\sum_{i=1}^{n-1} a_i x_i + a_n x_n) \]
        \[ = f((1-a_n)\sum_{i=1}^{n-1} \frac{a_i}{1-a_n} x_i + a_n x_n) \]
        We recognize $a_n$ and $1-a_n$ as a convex combination, and so, 
        \[ \leq a_n f(x_n) + (1-a_n) f(\sum_{i=1}^{n-1} \frac{a_i}{1-a_n} x_i) \] 
        We now recognize $\frac{a_1}{1-a_n}... \frac{a_{n-1}}{1-a_n}$ as a convex 
        combination. And so, 
        \[ \leq a_n f(x_n) + (1-a_n)\sum_{i=1}^{n-1} \frac{a_i}{1-a_n}f(x_i) = \sum_{i=1}^n a_i f(x_i) \]
        This proves the lemma.

    \end{proof}

    \begin{corollary}\label{cor:cool_log_property}
        If $a_1..a_n$ is a convex combination and $x_1...x_n > 0$, then, 
        \[ \log(a_1x_1 + .. a_nx_n) \geq a_1\log(x_1) + ... + a_n\log(x_n) \]
    \end{corollary}
    \begin{proof}
        This follows from the convexity of $-\log$.
    \end{proof}

    Another fact about convex functions, which we will not prove, is that they are continuous.

    \subsection{Jensen's Inequality}

    Note that there is some sense in which a probability distribution provides a sort of 
    generalized convex combination. From the above, it follows that if $X$ is a discrete random variable with 
    state space $\Omega = \{x_1...x_n\}$, then: 

    \[ \mathbb E[f(X)] = \sum_{i=1}^n f(x_i) \mathbb P\{X = x_i\} \]
    \[ f(\mathbb E(X)) = f\bigg(\sum_{i=1}^n x_i \mathbb P\{X = x_i\} \bigg) \]

    Note that we recognize $\mathbb P\{X = x_1\}... \mathbb P\{X=x_n\}$ as a convex combination. 
    And thus, 

    \[ f(\mathbb E[X]) \leq \mathbb E[f(X)]  \]

    The same is true more generally. 


    \begin{theorem}[Jensen's Inequality]
        Let $(\R, \Borel, \mathbb P)$ be a probability space and 
        $f$ be a convex function. Then, $f(\mathbb E[X]) \leq \mathbb E[f(X)]$. 
    \end{theorem}

    \begin{proof}
        
        Bear in mind the finite case provided earlier. Intuiviely, you might imagine that 
        the more general case is true by continuity. But to offer a reasonably concise proof, 
        we will use another property of convex functions. A convex function $f$ has, at each $x$, 
        a support line $\ell_x$ such that $\ell_x(x) = f(x)$, and $\ell_x \leq f$ everywhere. 
        So let $x_0 = \mathbb E[X]$ and let $\ell$ be the support line at $x_0$. If we let 
        $\ell(x) = ax + b$, it's clear that $\mathbb E[\ell(X)] = a \mathbb E[X] + b = \ell(x_0)$. 
        But by the decreasing property, $\mathbb E[\ell(X)] \leq \mathbb E[f(X)]$. Putting this all 
        together, we find,

        \[ f(\mathbb E[X]) = f(x_0) = \ell(x_0) = \mathbb E[\ell(X)] \leq \mathbb E[f(X)] \]

    \end{proof}

    \subsection{Hölder's Inequality}

    Conventional proofs of Hölder's Inequality typically rely 
    on Young's Inequality. While these are generally faster, I think they 
    are slightly less intuitive. Let us actually prove a general statement, 
    from which Hölder's Inequality is a corollary. 

    \begin{theorem}
        Let $X_1...X_n$ be nonnegative measurable functions and 
        $a_1...a_n$ be a convex combination. Then, 

        \[ \int\bigg(\prod_{i=1}^n X_i^{a_i}\bigg)d\mu \leq \prod_{i=1}^n \bigg(\int X_i d\mu \bigg)^{a_i}  \]
    \end{theorem}

    \begin{proof}
        It is simple to check that if $\bigg(\int X_i d\mu \bigg) = 0$ or $\infty$ for any 
        $i$, then the corresponding inequality is trivial. Otherwise, assume all relevant integrals 
        are finite and nonzero. We will first reduce the inequality to something simpler

        \[ \int \prod_{i=1}^n \bigg(\frac{X_i}{\int X_i d\mu} \bigg)^{a_i}d\mu  \leq 1 \]

        So define $Y_i = \frac{X_i}{\int X_i d\mu}$. Clearly, each $Y_i$ integrates 
        to $1$. It remains to show that $\int \prod_{i=1}^n Y_i^{a_i} d\mu \leq 1$. Indeed, 
        we can check that, by the Corollary ~\ref{cor:cool_log_property},
        
        \[ \prod_{i=1}^n Y_i^{a_i} = \exp\bigg(\log\bigg(\prod_{i=1}^n Y_i^{a_i}\bigg) \bigg)= \exp\bigg( \sum_{i=1}^n a_i \log(Y_i) \bigg) \]
        \[ \leq \exp\bigg(\log\bigg(\sum_{i=1}^n a_i Y_i \bigg)\bigg) = \sum_{i=1}^n a_i Y_i \]

        The above is only rigorous when $\prod_{i=1}^n Y_i^{a_i} > 0$, but the inequality still 
        holds trivially when $\prod_{i=1}^n Y_i^{a_i} = 0$. And so, by 
        monotonicity, 

        \[ \int \prod_{i=1}^n Y_i^{a_i} d\mu \leq \int \sum_{i=1}^n a_i Y_i d\mu = \sum_{i=1}^n a_i \int Y_i d\mu = 1 \]

        Which completes the proof.

    \end{proof}

    \begin{corollary}[Hölder's Inequality]
        If $p,q > 0$ s.t $1/p + 1/q = 1$, then for all measurable 
        $f$ and $g$, 

        \[ \int |fg|d\mu \leq \bigg( \int |f|^p  d\mu  \bigg)^{1/p}\bigg( \int |g|^q d\mu  \bigg)^{1/q} \]
    \end{corollary}

    \begin{proof}
        Let $X_1 = |f|^{p}$ and $X_2 = |g|^{q}$.
    \end{proof}

    \begin{corollary}[The Cauchy-Schwarz Inequality]
        If $f$ and $g$ are measurable, then, 

        \[ \int |fg|d\mu \leq \bigg( \int f^2  d\mu  \bigg)^{1/2}\bigg( \int g^2 d\mu  \bigg)^{1/2} \]

    \end{corollary}

    \begin{proof}
        Apply Hölder's Inequality with $p = q = 2$.
    \end{proof}

    \section{Hilbert Spaces and $\El^p$ Spaces}

    The theory of Hilbert Spaces is quite general and far-reaching. A Hilbert space can 
    be thought of as a generalization of Euclidean space, which includes the vector space 
    $\mathbb R^n$ and the Euclidean dot product. More generally, a Hilbert Space is a vector 
    space $\mathcal H$ with an inner product $\langle \cdot , \cdot \rangle$ which induces a 
    complete metric space. Recall an inner product is a map $\mathcal H \times \mathcal H \to \mathbb C$ which 
    satisfies, 

    \begin{itemize}
        \item $\langle f, g \rangle = \overline{\langle g,f \rangle}$
        \item $\langle af_1 + bf_2, g \rangle = a \langle f_1, g \rangle + b \langle f_2, g \rangle$
        \item $\langle f, f \rangle \geq 0$
    \end{itemize}

    Where $\bar{\cdot}$ denotes complex conjugation. For our purposes though, we can ignore 
    this distinction. If the inner product is a map to $\mathbb R$, we find it is symmetric and linear 
    in both its arguments. The norm induced by the inner product is: $\|f\| = \sqrt{\langle f,f \rangle}$. 

    \subsection{The $\El^p$ Spaces}

    Recall from our study of 
    measurable functions that linear combinations of measurable functions 
    are measurable. It is not hard to see from this that measurable functions constitute 
    a vector space. In this space, we define the norm,

    \[ \|X\|_p = \bigg(\int X^p d\mu \bigg)^{1/p} \]

    We show that this operation does induce a valid norm, and we construct the corresponding 
    Hilbert space. We now prove the triangle inequality, which is a crucial step to verifying that 
    we have a valid norm on our hands. A first step to realize is that, 
    in the language of norms, Hölder's Inequality states that $|XY| \leq \|X\|_p\|Y\|_q$.

    \begin{theorem}[Minkowski's Inequality]
        The triangle equality holds in the norm $\|\cdot\|_p$: 

        \[ \|X + Y\|_p < \|X\|_p + \|Y\|_p \] 

    \end{theorem}

    \begin{proof}
        We proceed in the usual fashion. For now, assume that 
        $0 < p < \infty$. The other cases are trivial. First, observe that 
        $f(x) = x^p$ is convex for $x \geq 0$. And thus, for any 
        $x,y$, $|x + y|^p = |\frac{1}{2}(2x) + \frac{1}{2}(2y)|^p \leq 
        \frac{1}{2}|2x|^p + \frac{1}{2}|2y|^p = 2^{p-1}(|x| + |y|)$. Note 
        the same will hold true using $1/p$.

        \[ \|X+Y\|_p = \bigg(\int |X + Y|^p d\mu \bigg)^{1/p} \leq 2^{\frac{p-1}{p}} \bigg( \int|X|^p + |Y|^p d\mu \bigg)^{1/p} \]
        \[ = 2^{1 - \frac{1}{p}} \bigg( \|X\|_p^p + \|Y\|_p^p \bigg)^{1/p} \leq 2^{1-\frac{1}{p}} \bigg( \frac{1}{2} (2\|X\|_p^p)^{1/p} + \frac{1}{2} (2\|Y\|_p^p)^{1/p} \bigg) \]
        \[ = 2^{1- \frac{1}{p}} \bigg(2^{\frac{1}{p} - 1}\|X\|_p + 2^{\frac{1}{p} - 1}\|Y\|_p  \bigg) = \|X\|_p + \|Y\|_p \]

        To now address the $p \in \{0,\infty\}$ cases, recall that, 

        \[ \|X\|_0 = \mu(\{|X| \geq 0 \}) \hspace{0.5cm} \|X\|_\infty = \text{ess-}\sup |X| \]

        Where the ess-sup is defined as the smallest supremum of all functions $Y$ equal to $X$ almost everywhere:

        \[ \text{ess-}\sup X = \inf_{Y = X \: a.e.}\sup Y \]
        
        So then, by subadditivity,

        \[ \|X + Y\|_0 = \mu(\{|X| \geq 0\} \cup \{|Y| \geq 0 \}) \] 
        \[ \leq \mu(\{|X| \geq 0\}) + \mu(\{|Y| \geq 0\}) = \|X\|_0 + \|Y\|_0 \]

        Now for $p = \infty$. Intuiviely, the idea is just that 
        $\max(X + Y) \leq \max(X) + \max(Y)$, but the almost-everywhereness 
        adds some subtleties. Indeed, note that for any 
        $X' = X$ a.e. and $Y' = Y$ almost everywhere, $X' + Y' = X + Y$ almost everywhere. 
        Furthermore, $\sup\{X' + Y'\} \leq \sup\{X'\} + \sup\{Y'\}$. So then, 

        \[ \|X+Y\|_\infty = \inf_{Z = X+Y \: a.e.}\sup |Z| \leq \inf_{X' = X, Y' = Y \: a.e.} |X' + Y'|  \]
        \[ \leq \inf_{X' = X \: a.e.}\sup |X| + \inf_{Y' = Y a.e.}\sup |Y| = \|X\|_\infty+ \|Y\|_\infty \]

        This proves the claim for all $p$.
    \end{proof}

    we define the space $\El^p(\Omega, \F, \mu)$ as $\El^p(\Omega, \F, \mu) = \{X \in \Mf : \|X\|_p < \infty\}$. Note it is possible for $\|X - Y\|_p = 0$ but for $X \neq Y$ everywhere. However, 
    it is necessary and sufficient that $X = Y$ \emph{almost everywhere}. And thus, 
    we think of $\|\cdot\|_p$ as operating on the equivalence classes of 
    functions, with equivalence if two functions are equivalent almost everywhere. Thus, 
    $\|\cdot\|_p$ is not truly a norm (rather it is a pseudonorm) since 
    $\|X-Y\| = 0$ does not imply $X = Y$. To make it a norm, we let 
    $L^p$ be the equivalence classes of $\El^p$.

    \begin{theorem}
        $\|X-Y\|_p = 0$ if and only if $X = Y$ almost everywhere.
    \end{theorem}
    \begin{proof}
        Let $A = \{X \neq Y\}$. Now let $A_0=\{|X-Y| \geq 1/2\}$ and $A_n = \{1/2^n > |X - Y| \geq 1/2^{n-1}\}$. It is clear to see that 
        $A = \bigcup_{i=0}^n A_n$ and also that all the $A_n$'s are disjoint. It therefore follows 
        from countable additivity that $\mu(A) = \sum_{i=1}^n \mu(A_n)$. Then if 
        If $\mu(A) > 0$, then there must be some $n^\star$ for which $\mu(A_{n^\star}) > 0$.
        Assume for now that $n^\star > 0$. It follows that, 

        \[ \|X-Y\|_p^p = \int_{A_{n^\star}} |X-Y|^pd\mu + \int_{A_{n^\star}^c} |X-Y|^p d\mu \geq \frac{1}{(2^{n^\star - 1})^p}\mu(A_{n^\star})  \]

        (Here, we use the notation that $\int_C Xd\mu = \int X \Ind\{C\}d\mu$). 
        The above expression is obviously nonzero. If $n^\star = 0$, a similar analysis 
        holds. Thus, it must be that $\mu(A) = 0$. This shows the ``only if''. For the ``if'', we have,

        \[ \|X-Y\|_p^p = \int |X-Y|^p d\mu =  \int_A |X-Y|^p d\mu + \int_{A^c} |X-Y|^p d\mu\]

        $\int_A |X-Y|^p d\mu = 0$ because $A$ is negligible. $\int_{A^c} |X-Y|^p d\mu = 0$ because 
        $X=Y$ everywhere in $A^c$. Thus the above is entirely nonzero. 
    \end{proof}

    From the above, it's clear that $\|\cdot\|_p$ is a valid norm acting on $L^p$. In order to verify that 
    $L^p$ is a Hilbert space, it is necessary to verify that $\|\cdot\|_p$ induces a complete metric. 

    \begin{theorem}
        $L^p$ is complete.
    \end{theorem}
    \begin{proof}
        Recall a space is complete if Cauchy sequences converge to a limit function. So let 
        $X_1, X_2...$ be a Cauchy sequence. Here's a first step. Let 
        $n_k$ be a sequence of numbers such that for $n,m \geq n_k, \|X_n - X_m\|_p < 1/2^k$. 
        It then follows that $\{X_{n_k}\}_k$ is a Cauchy sequence and $\sum_{k=1}^\infty \|X_{n_{k+1}} - X_{n_k}\|_p \leq 1$. Now, define $Y_k = \inf_{m\leq k}X_{n_m}$. 


        \begin{Proposition} 
            $\lim\inf_k X_{n_k} = \lim\sup_k X_{n_k}$
        \end{Proposition}
        \begin{proof}
            Define $L_{i,j} = \inf_{i \leq k \leq j}X_{n_k}$. First observe that, 

            \[ \|X_{n_i} - L_{i,j}\|_p = \bigg \|(X_{n_i} - Y_{i,j}) \sum_{k=i}^j \Ind\{Y_{i,j} = X_{n_k}\}\bigg \|_p  \]
            \[ \leq \sum_{m=i}^k \bigg\| (X_{n_i} - Y_{i,j})\Ind\{Y_{i,j} = X_{n_k}\}\bigg\|_p \leq \sum_{m=i}^k \| X_{n_i} - X_{n_k}\|_p < \sum_{k=i}^j \frac{1}{2^i} \leq 2^{i-1}  \]

            Now, taking $j \to \infty$, we find $Y_{i,j} \to \inf_{k > i} X_{n_k}$ and thus $\|X_{n_i} - \inf_{k > i X_{n_k}}\|_p \leq 2^{i-1}$. 
            A similar argument yields $\|X_{n_i} - \sup_{k > i X_{n_k}}\|_p \leq \leq \frac{1}{2^{i-1}}$. 
            It follows that, 

            \[ \|\lim\inf_{k \geq i}X_{n_k} - \lim\sup_{k \geq i}X_{n_k}\|_p \leq \|\inf_{k \geq i}X_{n_k} - \sup_{k \geq i}X_{n_k}\|_p \]
            \[\leq \|\inf_{k \geq i}X_{n_k} - X_{n_i}\|_p + \|\sup_{k \geq i}X_{n_k} - X_{n_i}\|_p \leq 2\bigg(\frac{1}{2^{i-1}}\bigg)  = \frac{1}{2^{i-2}} \]

            Observe that, for all $i$, $\lim\inf_{k \geq i}X_{n_k} = \lim\inf_{k \geq 1}X_{n_k}$ and likewise for the 
            $\lim\sup$. It thus follows that $\|\lim\sup_{k}X_{n_k} - \lim\inf_k X_{n_k}\|_p < \frac{1}{2^{i-2}}$ for all $i$, and thus, 
            $\lim\sup_{k}X_{n_k} = \lim\inf_k X_{n_k}$ almost everywhere.
        \end{proof}
        
        It follows that $X_{n_k}$ converges to a limit; call it $X_\infty$. We 
        will now show that $X_{n} \to X_\infty$ as well. This follows as a simple 
        consequence of the triangle inequality and the definition of a Cauchy sequence. 
        Indeed, take $M$ such that $\|X_n - X_m\|_p < \epsilon / 2$ for all $n,m \geq M$. Now let 
        $K$ be such that $k \geq K$ implies that $\|X_{n_k} - X_\infty\|_p < \epsilon / 2$. Letting 
        $N = \max(N,n_K)$, it follows that for all $n \geq N$,

        \[ \|X_n - X_\infty\|_p \leq \|X_n - X_{n_K}\|_p + \|X_{n_K} - X_\infty\|_p \leq \epsilon/2 + \epsilon/2 = \epsilon \]

        Taking $\epsilon \to 0$, we achieve the desired result.

    \end{proof}

    \section{Convergence Notions}

    The last topic in this chapter is what it means, exactly, for one random variable 
    to converge to another. We have just seen one: convergence in $L^p$. There are two 
    more we must consider: \emph{convergence in probability} and \emph{almost sure convergence}. 

    \begin{definition}[Convergence in Probability]
        We say $X_n \to X$ in probability if, for all $\epsilon > 0$, $\mathbb P(\{|X_n - X| > \epsilon\}) \to 0$
    \end{definition}

    \begin{definition}[Almost Sure Convergence]
        We say $X_n \to X$ almost surely if $\Prob(\{\lim_n X_n = X\}) = 1$
    \end{definition}

    \begin{definition}[Convergence in $L^1$]
        We say $X_n \to X$ in $L^1$ if $\|X_n - X\|_1 \to 0$
    \end{definition}

    \begin{Proposition} 
        Almost sure convergence implies convergence in probability
    \end{Proposition}

    \begin{proof}
        We first show the statement about convergence in probability. Fix $\epsilon > 0$. 
        First, we know that $\lim_n X_n$ exists almost surely. Simply write,

        \[ 1 = \mathbb P(\lim_n X_n = X) = \mathbb P(\{\lim_n |X_n - X| = 0\}) \]
        \[ \leq \mathbb P(\{\lim_n |X_n - X| < \epsilon \}) \leq \mathbb P(\lim_n\{ |X_n - X| < \epsilon \}) \]
        \[ = \lim_n \Prob(\{|X_n - X| < \epsilon\}) \]

        Where the last line is due to dominated convergence.
    \end{proof}

    \begin{theorem}
        Convergence in probability implies convergence in $L^1$.
    \end{theorem}
    \begin{proof}
        Suppose $X_n \to X$ in probability. Fix $\epsilon, \delta > 0$. Let 
        $N$ be such that $n \geq N \implies \mathbb P(\{|X_n - X| > \epsilon\}) < \delta$. 
        Now observe that for $n \geq N$,

        \[ \mathbb P(|X_n - X|) = \mathbb P(|X_n - X|\{|X_n - X| > \epsilon \} + |X_n - X|\{|X_n - X| \leq \epsilon \})\]
        \[ \leq \epsilon + \mathbb P(|X_n - X|\{|X_n - X| > \epsilon \}) \]

    \end{proof}