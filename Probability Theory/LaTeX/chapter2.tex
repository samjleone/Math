
\chapter{Extending Lebesgue-Stieltjes Measures \& Carathedory's Theorem}

Note that if you know you have a measure, you can do everything your heart desires! 
This chapter it not dedicated to the existence of measures so much as proving that
we can construct measures with desireable properties.

\section{The Problem}

You are a contractor. Your client comes to you, embarassed, and says \emph{hey, I have my 
Lebesgue-Stieltjes measure $\mu$. Could you help me define it uniquely 
 over all of $\Borel$}? You say \emph{sure thing, let me just plug it into my 
 measure extender machine}. And out pops a new measure consistent, defined over 
 new sets, and it is consistent with the old one. The main example to bear in mind is 
 the Lebesgue measure. \\ 

 Begin by saying that you have a measure $\mu_S$ (S for Start) defined on an \emph{semi-ring} $\mathcal A$. While 
 we have not yet defined semi-ring, think of it as an incredibly simple family of sets. For example, intervals of the form 
 $(a,b]$ comprise a semi-ring; we could specify a Lebesgue-Stieltjes function on this semi-ring.
 $d$-dimensional boxes like $\bigtimes_{i=1}^d (a_i,b_i]$ also comprise a semi-ring. The problem is basically this: 
 your customer comes to you and says \emph{hey buddy, I already know what I want $\mu_S$ to be like on 
 $\mathcal A$, can you help me out on $\sigma(\mathcal A)$?} You say, probably! More formally, we 
 seek to prove a theorem roughly of the form: 

 \begin{theorem}
   Under assumptions, given a baby measure $\mu_S$ defined on a semi-ring 
   $\mathcal A$, there exists a unique measure $\mu$ defined on $\sigma(\mathcal A)$ 
   which respects $\mu_S$.
 \end{theorem}

 \section{Semi-Rings \& Rings}

 This will be boring, but necessary. We are going to define a series 
 of related families of sets. Let $\Omega$ be the universe and 
 $\mathcal A$ be a family of subsets of $\Omega$. The relationships 
 to bear in mind are: 

 \[ \text{Semi-Ring} \implies \text{Ring} \implies \text{Algebra} \implies \sigma-\text{Algebra} \]

 \begin{definition}[Semi-Ring]
    A family of sets $\mathcal A$ is said to be a semi-ring if, for all 
    $A, B \in \mathcal A$,
    \begin{itemize}
        \item $\emptyset \in \mathcal A$
        \item $A \cap B \in \mathcal A$
        \item $A \setminus B = \cup_{1 \leq j \leq n}C_j$, where $C_j \in \mathcal A$ and the $C_j$'s 
         are all pairwise disjoint.
    \end{itemize}
 \end{definition}

 \begin{Proposition}
   $\mathcal I$ is a semi-ring.
 \end{Proposition}

 The canonical example to bear in mind is the semi-ring of half-open intervals. Define 
 $\mathcal I$ to be all the sets of the form $(a,b]$ with $a \leq b$ in $\mathbb R$. So 
 $\mathcal I = \{(a,b] : a,b \in \mathbb R\}$.

 \begin{proof}
   Letting $a = b$, $(a,b] = \emptyset$. Properties (ii) and (iii) can be checked by simple 
   casework on any two intervals $A = (a,b], B = (c,d]$.
 \end{proof}

 Note that semi-rings, in particular this semi-ring, 
 is not closed under union. If we add this property, we obtain a ring. 

 \begin{definition}[Ring]
   A family of sets $\mathcal A$ is said to be a ring if, for all 
   $A, B \in \mathcal A$,
   \begin{itemize}
       \item $\emptyset \in \mathcal A$
       \item $A \cup B$
       \item $A \setminus B \in \mathcal A$
   \end{itemize}
\end{definition}

Just like how we defined the $\sigma$-algebra generated by a set, the ring generated 
by $\mathcal A$ is the smallest ring containing $\mathcal A$.

\begin{Proposition}\label{prop:make_ring}
   Let $\mathcal A$ be a semi-ring. Let $\mathcal B$ be the set of finite disjoint unions 
   of elements of $\mathcal A$. Then $\mathcal B = $ ring($\mathcal A$)
\end{Proposition}

\begin{proof}
   We begin by showing $\mathcal B$ is a ring. Note that $\emptyset \in \mathcal B$ as $\emptyset \in \mathcal A$, so it can be considered 
   as the union of one element of $\mathcal A$. Now, write $A = \bigcup_{i=1}^n A_i, B = \bigcup_{j=1}^m B_j$. 
   We shall show that $A \cup B \in \mathcal A$. Indeed, $A \cup B = \bigcup_{i =1}^n A_i \cup \bigcup_{j=1}^m B_j$, which 
   is also an element of $\mathcal A$ by definition. Now, it remains to show that 
   $A \setminus B \in \mathcal A$. To see this, note that, 

   \[ A \setminus B = \bigcup_{i=1}^n A_i \setminus \bigcup_{j=1}^m B_j \]

   This can be understood as those elements $x$ which belong to at least one $A_i$, but 
   not a single $B_j$. From this, it's clear that this can be understood as: 

   \[ \bigcup_{i=1}^n \bigcap_{j=1}^m A_i \setminus B_j \] 

   Note also that by definition of a semi-ring, we can write $A_i \setminus B_j = 
   \cup_{k=1}^{n_{i,j}}C_{i,j,k}$, where these are all disjoint. And thus, we have, 

   \[  A \setminus B = \bigcup_{i=1}^n \bigcap_{j=1}^m \bigcup_{k=1}^{n_{i,j}} C_{i,j,k} \]

   Now there's a bit of a subtle thing going on. As the $C_{i,j,k}$'s are pairwise 
   disjoint for fixed $k$, if $x$ is in $\bigcap_{j=1}^m \bigcup_{k=1}^{n_{i,j}} C_{i,j,k}$, it 
   belongs to precisely one $C_{i,j,k}$ for each $j$. Let $\mathcal C_{i,j} = \{C_{i,j,k} : 1 \leq k \leq n_{i,j}\}$. 
   We may then write: 

   \[ \bigcap_{j=1}^m \bigcup_{k=1}^{n_{i,j}} C_{i,j,k} = \underbrace{\bigcup_{C_{1} \in \mathcal C_{i,1} ... C_{i,m} \in \mathcal C_{i,m} } C_1 \cap C_2... \cap C_m}_{ \in \mathcal B}  \]

   Since $\mathcal A$ is a semi-ring and thus is closed under intersection, 
   each $C_1 \cap C_2.. \cap C_m$ is in $\mathcal A$, so the union of such intersections is 
   in $\mathcal B$. Thus, 
   $\bigcap_{j=1}^m \bigcup_{k=1}^{n_{i,j}} C_{i,j,k}$ is in $\mathcal A$ for all $i$. 
   And so, as we've already shown $\mathcal B$ is closed under union, $A \setminus B \in \mathcal B$. 
   This proves the desired property. And so $\mathcal B$ is indeed a ring. \\ 

   To see that $\mathcal B = \text{ring}(\mathcal A)$ is simple. As $\text{ring}(\mathcal A)$ is a ring, 
   it must contain all unions of elements of $\mathcal A$, so $A \subseteq \mathcal B \subseteq \text{ring}(\mathcal A)$. 
   Taking the ring of all sides, and noting $\text{ring}(\mathcal B) = \mathcal B$, as $\mathcal B$ is a ring, 
   $\text{ring}(\mathcal A) \subseteq \mathcal B \subseteq \text{ring}(\mathcal A)$, so 
   $\mathcal B =\text{ring}(\mathcal A)$.

\end{proof}

\section{Extending Lebesgue-Stieltjes Measures from Semi-Rings to Rings}

We now show that we can extend measures from semi-rings to rings. First, we define a 
relaxed version of a measure that we care about. 

\begin{definition}[finitely additive measure]
   Suppose $\mu : \mathcal A \to \mathbb R$ is a function on subsets of 
   $\Omega$. We say $\mu$ is finitely additive if,
   \begin{itemize}
      \item $\mu(\emptyset) = 0$
      \item If $A \subseteq B, \mu(A) \leq \mu(B)$
      \item If $A,B \in \mathcal A$, $A \cap B = \emptyset$ and $A \cup B \in \mathcal A$, then $\mu(A \cup B) = \mu(A) + \mu(B)$
   \end{itemize}
\end{definition}

\begin{theorem}
   Let $\mu_S$ be a countably additive measure defined on a semi-ring $\mathcal A$. Let 
   $\mathcal B = $ ring($\mathcal A$). Then there exists a unique finitely additive measure 
   $\mu$ acting on $\mathcal B$ which respects $\mu_S$ over $\mathcal A$.
\end{theorem}

\begin{proof}
   We provide an explicit construction for $\mu$, then verify that 
   all is well. For an arbitrary element $B \in \mathcal B$, let $B = \cup_{i=1}^n A_i$ 
   (we know from Proposition ~\ref{prop:make_ring} that this is the form of such elements). 
   And assume without loss of generality that the $A_i$'s are disjoint. Why can we do this? 
   Note for $A,B \in \mathcal A$, we have $A \cup B = B \cup (A \setminus B) = B \cup \cup_{i=1}^m C_i$, 
   where the $C_i$'s are pairwise disjoint. By induction, then, every finite union can be represented 
   as a finite disjoint union. We do the perfectly natural thing: we want our measure to be 
   finitely additive, so our hand is forced. We define,
   
   \[ \mu(B) = \sum_{i=1}^n \mu_S(A_i) \]

   First, we must verify that this is consistent. Suppose $B = \cup_{j=1}^m B_i$. Then observe, 

   \[ \sum_{i=1}^n \mu_S(A_i) = \sum_{i=1}^n \mu_S(A_i \cap B) = \sum_{i=1}^n \mu_S(A_i \cap \bigcup_{j=1}^m B_j)  \]
   \[ = \sum_{i=1}^n \sum_{j=1}^m \mu_S(A_i \cap B_j) = \sum_{j=1}^m \mu_S(B_j) \]

   Where we have employed the assumption that $\mu_S$ is finitely additive over 
   $\mathcal A$ and utilized the fact that semi-rings are closed under intersection 
   (and so $\mu_S(A_i \cap B_j)$ is defined). From here, it is trivial to verify that 
   $\mu$ is finitely additive. Indeed, we simply seek to show that if 
   $A,B \in \mathcal B$ are disjoint, then $\mu(A \cup B) = \mu(A) \cup \mu(B)$. 
   First, write $A = \cup_{i=1}^n A_i, B = \cup_{j=1}^m B_j$. Again, assume the 
   $A_i$'s are pairwise disjoint, as are the $B_j$'s. But since $A \cap B  = 
   \emptyset$, the $A_i$'s are also disjoint with the $B_j$'s. So then, 

   \[ \mu(A \cup B) = \mu\bigg( \bigcup_{i=1}^n A_i \cup \bigcup_{j=1}^m B_j \bigg) = \sum_{i=1}^n \mu(A_i) + \sum_{j=1}^m \mu(B_j) = \mu(A) + \mu(B) \]

   Which proves the additivity property. By induction, one can easily establish that if 
   $B_1,B_2...B_n \in \mathcal B$ are all pairwise disjoint, then $\mu(\cup_{i=1}^n B_i) = \sum_{i=1}^n \mu(B_i)$. \\

   Uniqueness of $\mu$ is trivial. If there is a second $\mu'$ that respects $\mu_S$, then countable 
   additivity forces $\mu = \mu'$ over everything in $\mathcal B$.

\end{proof}

At this point, we would like to show that $\mu$ is not only finitely additive, but 
countably additive over $\mathcal B$. The following proposition ensures that countable 
additivity of $\mu_S$ over $\mathcal A$ is sufficient.

\begin{theorem}
   Letting $\mathcal A$ be a semi-ring and let $\mathcal J$ be the ring generated by $\mathcal A$. Let
   $\mu_S$, $\mu$ as described in the above theorem. Then if $\mu$ is countably additive 
   over $\mathcal I$, then $\mu$ is countably additive over $\mathcal J$.
\end{theorem}

\begin{proof}

   First, as $B \in \mathcal B$, we may write $B = \cup_{j=1}^m A_j$, where the $A_j$'s are pairwise 
   disjoint. So let us first restrict our analysis to a particular $A_j$. Note that 
   $B = \cup_{i=1}^\infty B_i$ as well. Note that, as the $A_j$'s are pairwise disjoint, as are 
   the $B_i$'s, it must be the case that each $A_j$ is a collection of the $B_i$'s. 
   So let $I_j$ be such that $A_j = \cup_{i \in I_j}B_i$. Note that, if we can prove that 
   $\mu(A_j) = \sum_{i \in I_j}\mu(B_i)$ for each $j$, we will be done, since then 
   finite additivity will imply,

   \[ \mu(B) = \sum_{j=1}^m \mu(A_j) = \sum_{j=1}^m \sum_{i \in I_j}\mu(B_i) = \sum_{i=1}^\infty \mu(B_i) \]

   So it remains to prove $\mu(A_j) = \sum_{i \in I_j}\mu(B_i)$ for arbitrary $A_j$. Finally, 
   observe that $B_i$ can be written as $\cup_{k=1}^{n_i}C_{i,k}$. And so, $A_j = \cup_{i \in I_j}\cup_{k=1}^{n_i}C_{i,k}$. 
   Then by countable additivity of $\mu_S$ on $\mathcal I$,

   \[ \sum_{i \in I_j} \mu(B_i) = \sum_{i \in I_j} \sum_{k=1}^{n_i}\mu_S(C_{i,k}) = \mu(A_j) \]


\end{proof} 

Now, at this point, it will prove somewhat difficult to prove theorems in full generality. 
So let us abandon our hope of working with completely arbitrary semi-rings. From now on, 
we will let $\mathcal I$ be the semi-ring of d-dimensional boxes 
$\{(a_1, b_1] \times (a_2, b_2] ... \times (a_d, b_d] : a_1,b_1...a_d,b_d \in \mathbb R\}$. 

\begin{Proposition}
   $\mathcal I$ as described is a semi-ring
\end{Proposition}

\begin{proof}
   Exercise
\end{proof}

It is worth asking: when is $\mu_S$ actually countably additive? Not any set function will do.
For instance, if $\mu_S(a,b] = 2^{b-a}$, even though this satisfies the monotonicity property and 
$\mu_S(\emptyset) = 0$, additivity crumbles. Thus, we will restrict our study to d-dimensional 
Lebesgue-Stieltjes measures. That is, we assume the existence of 
$d$ distribution functions $F_1... F_d$, and set, 

\[ \mu_S\bigg( \bigtimes_{i=1}^d (a_i,b_i] \bigg)  = \prod_{i=1}^d (F_i(b_i) - F_i(a_i))\]

\begin{theorem}
   $\mu$ as described is countably additive over $\mathcal I$ and thus its extension to 
   $\mathcal J$ is countably additive as well.
\end{theorem}

\begin{proof}
   Assume $A = (a_1,b_1] \times ... \times (a_d,b_d] \in \mathcal I$. Also 
   assume we have $A = \bigcup_{i=1}^\infty A_i$, where each 
   $A_i = \bigtimes_{j=1}^d (a_{i,j}, b_{i,j}]$. We seek to show, 

   \[ \mu_S(A) = \sum_{i=1}^n \mu_S(A_i) \]

   We will argue this via induction on $d$. \\ 
   
   \textbf{Base Case:} First, suppose $d=1$, so we consider 
   a measure on the real line. First, observe $\mu_S$ is then finitely additive. If 
   $(a,b]$ and $(c,d]$ are disjoint (assume $a < c$), then for $(a,b] \cup (c,d] \in \mathcal I$, we have 
   $b = c$, so $\mu_S((a,b] \cup (c,d]) = \mu_S((a,d]) = F_1(d) - F_1(a)$. Likewise, 
   $\mu_S((a,b]) + \mu_S((c,d]) = F_1(d) - F_1(c) + F_1(b) - F_1(a) = F_1(d) - F_1(a)$. 
   We use this finite additivity over and over again. Note indeed, that 

   \[ \sum_{i=1}^n \mu(A_i) = \mu\bigg(\bigcup_{i=1}^n A_i\bigg) \leq \mu\bigg(\bigcup_{i=1}^\infty A_i\bigg) = \mu(A) \]
   
   Taking $n \to \infty$, we find $\sum_{i=1}^\infty \mu(A_i) \leq \mu(A)$. We now seek to show 
   the reverse inequality. Fix any
   $\epsilon > 0$ and consider an augmentation of the $A_i$'s. Let 
   $\delta > 0$ and $\delta_1,\delta_2.... > 0$ be arbitrary for now.
   $A_i = (a_i, b_i]$, let $A_i' = (a_i, b_i + \delta_i]$. Also consider, 
   where $A = (a,b]$, consider the new interval $A' = [a+\delta, b]$. As 
   $\{A_i'\}_i$ provides a covering of $A$, it provides a covering of
   $A'$. Thus, it is possible to extract a finite cover. So let $I \subseteq \mathbb N$ be such that 
   $A' \subseteq \cup_{i \in I}A_i'$. 

   \begin{Proposition}
      If $A, B \in \mathcal B$, then $\mu(A \cup B) \leq \mu(A) + \mu(B)$
   \end{Proposition}

   \begin{proof}
      By additivity, 

      \[ \mu(A \cup B) = \mu(A) + \mu(B \setminus A) \leq \mu(A) + \mu(B)  \]

      By induction, this holds for any finite number of sets as well.
   \end{proof}
   
   We can use this fact to upper bound $\mu(A')$ by a nondisjoint union:

   \[ \mu(A') \leq \mu\bigg(\bigcup_{i\in I}A_i' \bigg) \leq \sum_{i \in I} \mu(A_i') \] 
   \[ = \sum_{i \in I} F(b_i + \delta_i) - F(a) = \sum_{i \in I} (F(b_i + \delta_i) - F(b_i) + F(b_i) - F(a)) \]
   \[ \leq \sum_{i =1}^\infty (F(b_i) - F(a_i)) + (F(b_i + \delta_i) - F(b_i)) \] 
   \[ = \sum_{i =1}^\infty \mu(A_i) + \sum_{i =1}^\infty (F(b_i + \delta_i) - F(b_i))  \]

   Additionally, 

   \[ \mu(A) = F(b) - F(a) = F(b) - F(a + \delta) + F(a + \delta) - F(a) = \mu(A') + F(a + \delta) - F(a) \] 


   By right continuity of $F$, we can let $\delta$ be such that $F(a + \delta) - F(a) < \epsilon / 2$. 
   We may also let each $\delta_i$ be such that $F(b + \delta_i) - F(b) < \epsilon / 2^i$. So then, 

   \[ \mu(A) < \mu(A') - \epsilon/2 \leq \sum_{i =1}^\infty \mu(A_i) + \sum_{i =1}^\infty \epsilon / 2^i - \epsilon / 2 \]
   \[ = \sum_{i =1}^\infty \mu(A_i) + \epsilon / 2 \]

   At last, taking $\epsilon \to 0$, we find $\mu(A) \leq \sum_{i =1}^\infty \mu(A_i)$. This 
   completes the proof of the base case.\\

   \textbf{Inductive Step:} Now, we assume that this theorem is true in 
   $d-1$ dimensions, and we seek to push it to $d$ dimensions. So suppose that 
   $A = \cup_{i=1}^\infty A_i$, where the $A_i$'s are disjoint. Now, without loss of generality, 
   suppose that we take a common refinemenet of the $A_i$'s in the $d$th dimension. That is, 
   let $H = \bigcup_{i=1}^n (a_{i,d} \cup b_{i,d})$ be the set of all numbers which are relevant 
   to our partition along dimension $d$. Now put $H$ in increasing order, such that: 
   $H = \{c_1 < c_2... < c_n\}$. So now, split the $A_i$'s by $H$ into a new collection 
   $A_1', A_2'...$. Each $A_i'$ can be written as $A_i' = \bigtimes_{i=1}^{d-1} (a_i', b_i'] \times (c_k, c_{k+1}]$ 
   for some $k$. Now, partition the $A_i'$'s into sets $I_1, I_2...$ such that 
   for $i \in I_k$, $A_i$'s dimension $d$ component looks like $(c_k, c_{k+1}]$. So then, 

   \[ A = \cup_{k=1}^\infty \cup_{i \in I_k}A_i' \]

   Define a new distribution function $G$ where $G(c_k) = \sum_{1 \leq j \leq k} \mu(\cup_{i \in I_k}A_i')$. 
   $G$ thus corresonds to a 1-dimensional distribution function. So by the base case, 
   we have,

   \[ \mu(A) = \sum_{k=1}^\infty \mu\bigg(\bigcup_{i \in I_k}A_i'\bigg) \]
   
   Now, let $\mu_{d-1}$ be the measure induced by considering the first $d-1$ dimensions of 
   the $A_i'$s. We have $\mu\bigg(\bigcup_{i \in I_k}A_i'\bigg) = (c_{k+1} - c_k) \mu_{d-1}\bigg(\bigcup_{i \in I_k}A_i'\bigg)$. 
   So then, by induction, 

   \[ \sum_{k=1}^\infty \mu\bigg(\bigcup_{i \in I_k}A_i'\bigg)  =  \sum_{k=1}^\infty  (c_{k+1} - c_k) \mu_{d-1}\bigg(\bigcup_{i \in I_k}A_i'\bigg)\]
   \[ = \sum_{k=1}^\infty  (c_{k+1} - c_k) \sum_{i \in I_k}\prod_{i=1}^{d-1} (a_i, b_i]  = \sum_{k=1}^\infty \sum_{i \in I_k}\prod_{i=1}^{d} (a_i, b_i] = \sum_{i=1}^\infty \mu(A_i') \] 

   Note that the same decomposition into the $d$th and first $d-1$ dimensions yields: 

   \[ \mu(A_i) = \sum_{i : A_i' \subseteq A_i} \mu(A_i') \]

   Which collectively implies that $\mu(A) = \sum_{i=1}^\infty \mu(A_i)$. This completes the inductive step and thus the whole proof.

\end{proof}

\begin{corollary}\label{cor:countably_additive_on_ring}
   If $\mu_S$ is a d-dimensional Lebesgue-Stieltjes measure on $\mathcal I$, 
   then $\mu$ is countably additive on $\mathcal J$
\end{corollary}

\subsection{Recap}

Let's pause for a moment to focus on what we've actually done. We have shown that 
if we have a $d$-dimensional distribution function, we can extend it to a 
countably additive measure on a ring. We will find that rings are very nice. In particular, 
rings can approximate sets in the Borel $\sigma$ algebra arbitrarily well. This is the fact 
we will use to define an outer measure.

\section{Outer Measures}

Thus far, we have worked our way ``up" from our semi-ring and tried to 
build up something more sophisticated on rings, a more complicated family of sets. 
Now, we will develop a sort of master function, called a \emph{outer measure}, which 
is indeed defined on all subsets of $\Omega$ and thus $\Boreld$ as well. 
While outer measures do not behave well in general, we will show that it acts 
nicely on $\Boreld$.\\ 

Recall we have a measure $\mu$ acting on the ring $\mathcal J$ generated by 
the half-open intervals. We define the following outer measure on subsets of 
$\mathbb R^d$: 

\[ \mu^\star(A) = \inf\{ \mu(J) : J \in \mathcal J, A \subseteq J \}\]

Intuitively, the idea is this: we wrap an element $J$ of 
$\mathcal J$ around $A$ as tightly as possible, and then take sizes
 the way we know how. Then like shrink wrap, we make $J$ as small as possible. 
 First, let us establish a key fact. 

 \begin{Proposition} \label{prop:consistent_with_ring}
   If $A \in \mathcal J$, then $\mu^\star(A) = \mu(A)$
 \end{Proposition}

 \begin{proof} 
   Obviously, as $A$ is a valid candidate from 
   $\mathcal J$, $\mu^\star(A) \leq \mu(A)$. Now we show the reverse. 
   Suppose by way of contradiction that $\mu^\star(A) < \mu(A)$. Then there 
   would exist a $B \in \mathcal J$ with $A \subseteq B$ such that $\mu(B) < \mu(A)$. 
   This is of course a contradiction of the monotonicity property.
 \end{proof}

 Now, we define a very general family of sets: the Lebesgue measurable sets. This will 
 turn out to be more general than we need.

 \begin{definition}[Lebesgue Measurable]\label{def:lebesgue_measurable}
   Say a set $E \subseteq \mathbb R$ is Lebesgue-measurable if it satisfies the 
   Caratheodory criterion: that for all 
   $A \subseteq \mathbb R$, $\mu^\star(A) = \mu^\star(A \cap E) + \mu^\star(A \cap E^c)$. 
   Let the Lebesgue measurable sets be $\mathcal L$.
 \end{definition}

As a first observation, note that proposition ~\ref{prop:consistent_with_ring} and 
Corollary ~\ref{cor:countably_additive_on_ring} collectively imply that $\mu^\star$ is 
countably additive on $\mathcal J$. Here are the remaining steps: 

\begin{enumerate}
   \item Prove that $\sigma(\mathcal J) = \Boreld$
   \item Observe $\mathcal J \subseteq \mathcal L$
   \item Prove $\mu^\star$ is countably additive on $\mathcal L$
   \item Prove $\mathcal L$ is a $\sigma$-algebra
   \item Deduce $\Boreld = \sigma(\mathcal J) \subseteq \sigma(\mathcal L) = \mathcal L$
\end{enumerate}

From this, it will follow that $\mu^\star$ is a countably additive measure on $\Boreld$, so 
we will be done. 

\subsection{Step 1}

\begin{theorem} 
   $\sigma(\mathcal J) = \Boreld$
\end{theorem}
\begin{proof}
   Let us first prove $\mathcal I \subseteq \Borel$ for $d=1$. Indeed, note, 

   \[ (a,b] = \cap_{n=1}^\infty (a, b+1/n) \]

   And so each $(a,b] \in \Borel$. Thus, $\mathcal I \subseteq \Borel$. It remains to show the reverse inclusion. 

   \begin{Proposition}
      Every open set in $\mathbb R$ can be written as the disjoint union of open intervals
   \end{Proposition}

   \begin{proof}
      Let $O$ be open. Let $O_Q = O \cap \mathbb Q$. Observe that by definition of opennes, for each 
      $q \in O_Q$, there exists a highest $\epsilon_q > 0$ such that $B_{\epsilon_q}(q) \subseteq O$. Now, let 
      $C = \{B_{\epsilon_q}(q) : q \in O_Q\}$ and $S = \cup_{I \in C}I = O$. I claim $O = S$. To see this, observe 
      for any $x \in O$ that there is an $\epsilon$ ball $B_\epsilon(x)$ contained in $O$. 
      If we let $q$ be a rational number s.t. $|x - q| < \epsilon/2$, by maximality of 
      $\epsilon_q$, we have $x \in B_{\epsilon_q}(q)$, so $x \in C$. Finally, let elements of $D$ be obtained by connecting 
      all intervals in $C$, so that $D$ consists of disjoint open intervals and $\cup_{I \in D}I = \cup_{I \in C}I = S = O$. 
      Thus, $O$ can be written as the disjoint open intervals provided in $D$.
   \end{proof}

   Thus, letting $O \in \mathcal G$ be some arbitrary open set in $\mathbb R$, where we know 
   $\mathcal G$ generates $\Boreld$. Thus, the set of open intervals, call it $\mathcal E$, generates 
   $\Borel$. Yet also, any interval $(a,b)$ can be written as:

   \[ (a,b) = \cup_{n = 1}^\infty (a,b - 1/n] \]

   Which is in $\sigma(\mathcal I)$. Thus, $\mathcal E \subseteq \sigma(\mathcal I)$, so 
   $\Borel = \sigma(\mathcal E) \subseteq \sigma(\mathcal I)$. This is sufficient to prove the claim 
   in dimension $d = 1$. It remains to prove it for higher dimensions. 

   \begin{Proposition} 
      Let $\mathcal E$ generate $\F$. Then 
      \[ \sigma(\{A_1 \times A_2... \times A_n : A_1... A_n \in \mathcal E\}) = \sigma(\{A_1 \times A_2... \times A_n : A_1... A_n \in \F\})  \]
   \end{Proposition}

   \begin{proof}
      Exercise. It is best to prove this when $d=2$ and proceed by induction.
   \end{proof}

   \begin{corollary} 
      If $\mathcal E_1, \mathcal E_2$ both generate $\F$, then, 
      \[ \sigma(\{A_1 \times A_2... \times A_n : A_1... A_n \in \mathcal E_1\}) = \sigma(\{A_1 \times A_2... \times A_n : A_1... A_n \in \F\})  \]
      \[ = \sigma(\{A_1 \times A_2... \times A_n : A_1... A_n \in \mathcal E_2\}) \]
   \end{corollary}

   \begin{Proposition} 
      Every open set in $\mathbb R^d$ can be written as the (not necessaarily disjoint) union of 
      countably many open rectangles
   \end{Proposition}

   \begin{proof}
      Following the same outline as before, use the fact that $\mathbb Q^d$ is dense in 
      $\mathbb R^d$ and the definition of the open sets. The reason we no longer have disjointness 
      is that the union of two connected open rectangles may not be an open rectangle in dimension 
      greater than $1$.
   \end{proof}

   A corollary of this is that $\Boreld$ is generated by the set of open rectangles.

   \begin{corollary} 
      $\sigma(\mathcal I) = \Boreld$ for all $d$. 
   \end{corollary}

   \begin{proof}
      Let $\mathcal E_1 = \{ (a,b] : a,b \in \mathbb R\}$, which is simply 
      $\mathcal I$ in dimension $1$. Also let $\mathcal E_2 = \{ (a,b) : a,b \in \mathbb R\}$.
      Then, 

      \[ \sigma(\mathcal I) = \sigma(\{I_1 \times I_2.. \times I_d : I_1,..I_d \in \mathcal E_1\}) \]
      \[ = \sigma(\{A_1 \times A_2... \times A_d : A_1..A_d \in \sigma(\mathcal E_2)\})  = \Boreld \]
   
   \end{proof}

   And of course, $\sigma(\mathcal I) = \sigma(\mathcal J)$. This concludes the proof.

\end{proof}

\subsection{Step 2}

\begin{theorem}
   $\mathcal J \subseteq \mathcal L$
\end{theorem}

\begin{proof} 
   Consider arbitrary $E \in \mathcal J$. Now consider arbitrary $A \subseteq \mathbb R^d$. We desire 
   to show that 

   \[ \mu^\star(A) = \mu^\star(A \cap E) + \mu^\star(A \cap E^c) \]

   We will prove this by showing the two corresponding inequalities. Let 
   $J_1,J_2$ be such that, 

   \[ \mu(J_1) < \mu^\star(A \cap E) + \epsilon/2 \] 
   \[ \mu(J_2) < \mu^\star(A \cap E^c) + \epsilon/2 \] 

   Let $J = J_1 \cup J_2$. It's clear that
   $J \in \mathcal J$. Furthermore, as $A \cap E \subseteq J, A \cap E^c \subseteq J, 
   A = (A \cap E) \cup (A \cap E^c) \subseteq J$. So then we have,

   \[ \mu^\star(A) \leq \mu^\star(J) \leq \mu(J_1) + \mu(J_2) < \mu^\star(A \cap E) + \mu^\star(A \cap E^c) + \epsilon \]

   Taking $\epsilon \to 0$, we have side of the equality. 
   Now, we show the reverse inequality. We seek to show, 

   \[  \mu^\star(A \cap E) + \mu^\star(A \cap E^c) \leq \mu^\star(A) \]

   Suppose that $A \subseteq J$. Then $A \cap E \subseteq J \cap E$. Furthermore, 
   $A = (A \cap E) \cup (A \cap E^c) \subseteq (J \cap E) \cup (J \cap E^c)$. And thus, 

   \[ \mu^\star(A) + \epsilon \geq \mu(J) = \mu(J \cap E) + \mu(J \cap E^c) \]

   But note that $J \cap E, J \cap E^c \in \mathcal J$, so, 

   \[ \geq \mu(A \cap E) + \mu(A \cap E^c) \]

   Taking $\epsilon \to 0$, we're done.

\end{proof}

\subsection{Step 3}

\begin{theorem}
   $\mu^\star$ is countably additive on $\mathcal L$
\end{theorem}

\begin{proof}

   First, observe that $\mu^\star(\emptyset) = \emptyset$ trivially, as 
   $\emptyset \in \mathcal J$ and $\mu(\emptyset) = 0$. Now, assume 
   $A, B \in \mathcal L$ with $A \subseteq B$. Note that 
   for any $J \in \mathcal J$ with $B \subseteq J$, $A \subseteq J$. And thus, 

   \[ \mu^\star(A) = \inf\{\mu(J) : A \subseteq J\} 
   \leq  \inf\{\mu(J) : B \subseteq J\} = \mu^\star(B)  \]

   It remains to verify that $\mu^\star$ is countably additive. Let us begin with finite additivity. 
   It remains to show that for any $A,B \in \mathcal L$ that $\mu^\star(A \cup B) = \mu^\star(A) + \mu^\star(B)$. 
   Fix $\epsilon > 0$.
   First, let $J_A, J_B \in \mathcal J$ be such that, $\mu(J_A) - \mu^\star(A) < \epsilon/2$ and likewise for 
   $J_B$. It then follows that, $J_A \cup J_B$ is a valid cover of $A \cup B$, and so: 

   \[ \mu^\star(A \cup B) \leq \mu(J_A \cup J_B) \leq \mu(J_A) + \mu(J_B)< \mu^\star(A) + \mu^\star(B) + \epsilon \]

   Taking $\epsilon \to 0$, it's clear that $\mu^\star(A \cup B) \leq \mu^\star(A) + \mu^\star(B)$. Now, we would 
   like the reverse inequality: $\mu^\star(A) + \mu^\star(B)\leq \mu^\star(A \cup B)$. To see this, suppose that 
   $J$ is such that $\mu(J) < \mu^\star(A \cup B) + \epsilon$. Consider any 
   $J_A, J_B$ s.t. $A \subseteq J_A$ and $B \subseteq J_B$. Now, let $J_B' = J_B \setminus J_A$. We still have $J_B' \supseteq B$. 
   So then,

   \[ \mu^\star(A) + \mu^\star(B) < \mu(J_A \cap J) + \mu(J_B' \cap J) \] 
   By additivity on the ring and monotonicity,
   \[  = \mu((J_A \cap J) \cup (J_B' \cap J))\leq \mu(J) \leq \mu^\star(A \cup B) + \epsilon \] 
   Taking $\epsilon \to 0$, we have proven finite additivity in the 
   $n=2$ case; the general finite case follows easily by induction.\\

   Now, we proceed to countable additivity. Suppose that 
   $A_1,A_2... \in \mathcal L$ are all disjoint. Let $A = \cup_i A_i$. First, observe by finite addivity and monotonicity that, 

   \[ \mu^\star(A) \geq \mu^\star(\bigcup_{i=1}^n A_i) = \sum_{i=1}^n \mu^\star(A_i) \]

   Taking $n \to \infty$, we have one inequality. It remains to show the opposite. Again, 
   fix $\epsilon > 0$. We can this by showing that, for all $\epsilon > 0$,

   \[ \mu^\star(A) \leq  \sum_{i=1}^\infty \mu^\star(A_i) + \epsilon  \]

   To see this, let $J_i$ be such that $\mu(J_i) < \mu^\star(A_i) + \epsilon/2^i$. 
   And let $A \subseteq J$. It follows that $A_i \subseteq J\cap J_i  \subseteq J_i$, so 
   $\mu(J \cap J_i) < \mu(A_i) + \epsilon/2^i$. And thus, by countable additivity on the ring 
   $\mathcal J$, we obtain,

   \[ \mu^\star(A) \leq \mu(J) = \sum_{i=1}^\infty \mu(J \cap J_i) \leq \sum_{i=1}^\infty \mu^\star(A_i) + \epsilon/2^i = \sum_{i=1}^\infty \mu^\star(A_i) + \epsilon \] 

   Taking $\epsilon \to 0$, we conclude the desired result.
\end{proof}

\subsection{Step 4}

\begin{theorem}
   $\mathcal L$ is a $\sigma$-algebra
\end{theorem}

\begin{proof}
   First, clearly $\emptyset \in \mathcal L$. Additionally, note that if 
   $E \in \mathcal L$, then, for all $A \subseteq \mathbb R^d$, we have, 

   \[ \mu^\star(A) = \mu^\star(A \cap E) + \mu^\star(A \cap E^c) \]

   Which also implies that $E^c$ is Lebesuge-measurable. Thus, $\mathcal L$ is closed 
   under complement. Let us now assume that $E_1,E_2.. \in \mathcal L$. We then have 
   that, letting $E = \cup_i E_i$,

   \[ \mu^\star(A) = \mu^\star(\cup_i A_i) = \sum_i \mu^\star(E_i) = \sum_i \mu^\star(A \cap E_i) + \mu^\star(A \cap E_i^c) \]
   \[ = \sum_i \mu^\star(A \cap E_i) + \sum_i \mu^\star(A \cap E_i^c) \]
   \[ = \mu^\star(\cup_i A \cap E_i) + \mu^\star(\cup_i A \cap E_i^c)  = \mu^\star(A \cap E) + \mu^\star(A \cap E^c) \] 
   
   And thus $E$ is Lebesgue measurable. This concludes the proof.
\end{proof}

\subsection{Step 5}

Now, we have that $J \subseteq \mathcal L$ and $\sigma(\mathcal J) = \Boreld$. 
And so, $\Boreld \subseteq \sigma(\mathcal J) \subseteq \sigma(\mathcal L) = \mathcal L$. 
And thus, $\Boreld \subseteq \mathcal L$. And since $\mu^\star$ is a countably additive measure on 
$\mathcal L$, we find that $\mu^\star$ is also a countably additive measure on 
$\Boreld$. This is the desired result.

\section{The $\pi$-$\lambda$ Theorem and Uniqueness}

We will define two families of sets, state \& prove the $\pi$-$\lambda$ theorem, 
and give an application w.r.t. the Lebesgue measure. 

\begin{definition}[$\pi$-System]
   Say $P$ is a $\pi$ system if it is closed under intersection
\end{definition}
\begin{definition}[$\lambda$-System] 
   Say $L$ is a $\lambda$ system if, 
   \begin{itemize}
      \item $\emptyset \in L$
      \item $L$ is closed under complement
      \item If $A_1..A_n \in L$ and the $A_i$'s are pairwise disjoint, then $\cup_{i=1}^n A_i \in L$
   \end{itemize}
\end{definition}
\begin{theorem}[The $\pi$-$\lambda$ Theorem]
   Say $P$ is a $\pi$ system contained in a $\lambda$ system $L$. Then 
   $\sigma(P) \subseteq L$. 
\end{theorem}
\begin{proof}
   We show that $\lambda(P)$, the smallest $\lambda$ system containing $P$, is a 
   $\sigma$ algebra. Thus, $\sigma(P) \subseteq \lambda(P) \subseteq L$, since 
   $L$ is already a $\lambda$ system. Thus, it remains to prove that $\lambda(P)$ is a 
   $\sigma$ algebra.

   \begin{Proposition}
      A family of sets which is a $\pi$ and $\lambda$ system is also a $\sigma$ algebra.
   \end{Proposition}

   \begin{proof}
      The closure properties of a $\sigma$ algebra can be easily checked
   \end{proof}

   Thus, it remains to show that $\lambda(P)$ is a $\pi$-system, i.e. it is closed under 
   intersection. 

   \begin{lemma}
      Let $L$ be a $\lambda$ system. For $A \in L$, let, 
      \[ L_A = \{B \in L : A \cap B \in L\} \]
      Then $L_A$ is a $\lambda$ system.
   \end{lemma}

   \begin{proof}
      Check the properties of a $\lambda$ system
   \end{proof}

   \begin{lemma}
      The intersections of a $\lambda$ system is a $\lambda$ system
   \end{lemma}

   \begin{proof}
      Check the properties of a $\lambda$ system (not hard)
   \end{proof}

   Consider the following set $G$:

   \[ G = \{A \in \lambda(P) \: s.t. \: A \cap E  \in \lambda(P) \:, \forall E \in P \} \]

   Obviously, 

   \[ G = \bigcap_{E \in P}(\lambda(P))_E \]

   Combining the above two lemmas, it follows that $G$ is a 
   $\lambda$ system. As $P$ is a $\pi$ system, 
   $P \subseteq G$, so $\lambda(P) \subseteq \lambda(G) \subseteq \lambda(P)$, and 
   thus $\lambda(P) = G$. Thus, $G = \lambda(P)$. \\ 

   Now, we work out a little more. Write, 

   \[ H = \{ A \in \lambda(P) : A \cap B \in \lambda(P), \forall B \in \lambda(P) \} \]

   Now we find that, 

   \[ H = \cap_{A \in \lambda(P)} (\lambda(P))_E \]

   And thus again, $H$ is a $\lambda$ system. We find that 
   $\lambda(P) = H$. But obviously, $H$ is a $\pi$ system, so 
   we are done.

\end{proof}

We will now show that the restriction of $\mu^\star$ to $\Boreld$ is the only thing we can do. 
Suppose there is a second measure $\mu'$ which respects $\mu$ over $\mathcal J$; we can show that 
$\mu' = \mu^\star$ over $\Boreld$. With one assumption: assume 
$\mu^\star$ and $\mu'$ are $\sigma$-finite. How will this work? We proceed in the following steps: 
\begin{itemize}
   \item Let $\mathcal D = \{A \in \Boreld : \mu^\star(A) = \mu'(A)\}$
   \item Argue $\mathcal D$ is a $\sigma$ algebra.
   \item Observe $\mathcal J \subseteq \mathcal D$
   \item Deduce $\Boreld = \sigma(\mathcal J) \subseteq \sigma(\mathcal D) = \mathcal D$
   \item Conclude $\mu^\star = \mu'$ over all of $\Boreld$.
\end{itemize}

The only real work here is to show that $\mathcal D$ is in fact a 
$\sigma$ algebra, as theother steps are self explanatory. Because 
$\mu$ is $\sigma$-finite, let us write that $\Omega = \cup_j B_j$, where 
$B_j \in \mathcal J$ is countable and $\mu(B_j)< \infty$ for all $j$.\\

We first do the proof for finite measures. First, note that we can consider the 
$\lambda$ system $\mathcal I$. Clearly, if $\mu^\star = \mu'$ over $\mathcal J$, the same 
holds true over $J$. Now, we show $\mathcal D$ is a $\lambda$ system. Clearly, 
$\emptyset \in \mathcal D$. Furthermore, $\mathcal D$ is closed under complement, since, 

\[ \mu^\star(A^c) = \mu^\star(\Omega) - \mu^\star(A) \]
\[ = \mu'(\Omega) - \mu'(A) = \mu'(A^c) \] 

As $\mu', \mu^\star$ are countably (and thus finitely) additive, $L$ is obviously 
closed under disjoint unions as well, as, 

\[ \mu^\star(\cup_{i=1}^n A_i) = \sum_{i=1}^n \mu^\star(A_i) = \sum_{i=1}^n \mu'(A_i) = \mu'(\cup_{i=1}^n A_i) \]

Thus, $\mathcal D$ is a $\lambda$ system. We conclude from the 
$\pi$ lambda theorem that $\Boreld = \sigma(\mathcal I) \subseteq \mathcal D$, so 
$\mu^\star = \mu'$ over $\Boreld$. \\ 

The general $\sigma$-finite is simple. Let $B_1,B_2..$ be disjoint in $\mathcal J$ with 
$\cup_i B_i = \Omega$ and $\mu_i'(B_i) = \mu^\star(B) < \infty$. Then define $\mu^\star_i(A) = \mu^\star(A \cap B_i), 
\mu'_i(A) = \mu'(A \cap B_i)$ for al $A \in \Boreld$. It follows that, 

\[ \mu' = \sum_i \mu'_i \hspace{0.5cm} \mu^\star = \sum_i \mu^\star_i \]

And since each $\mu'_i, \mu^\star_i$ is a finite measure, by our prior work, they must agree. 
And so, all of $\mu', \mu^\star$ agree. This proves the unqiqueness, as desired!

\section{Consequences}

We should pat ourselves on the back and say.... whew. We are basically done with the hard work. 
For example, we can now define probability measures to our heart's content! For example, if we say, 

\[ \mu((a_1, b_1] ... \times (a_d, b_d]) =  \int_{a_1}^{b_1}.. \int_{a_d}^{b_d}\bigg(\frac{1}{2\pi}\bigg)^{-d/2} \prod_{i=1}^d e^{-\frac{1}{2}x_i^2}dx_i \]

Then we know $\mu$ is the unique measure on $\Boreld$ corresponding to the normal distribution!