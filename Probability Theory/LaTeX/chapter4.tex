\chapter{Densities \& The Radon Nikodym Theorem}

In your ordinary probability class, you learned about 
``densities'' as being a sort of continuous version of a Probability 
Mass Function. It was a sort of way to resolve the paradoxical fact that 
any outcome in a continuous distribution has zero probability, yet a point 
does have a notion of likelihood. If your random variable $X$ has a distribution 
function $F$, the density $f$ is defined as $f(t) = F(t)$. The density then had 
the nice property that $\Prob((-\infty, a]) = \int_{-\infty}^a f(t)dt$, where this holds by 
the fundamental theorem of calculus. Why are densities useful? Because it let's us do 
interesting things! We can easily calculate the expected value of $X$ by computing 
$\mathbb E[X] = \int tf(t)dt$. If $g(X)$ were, say, some other function, we calculate 
$\mathbb E[g(X)] = \int g(t)f(t)dt$. At the time, this is perhaps how you learned 
to \emph{define} the expectation of a random variable. Rather, we show this is an 
incredibly particular case of general densities between probability distributions. In this case, 
$f$ is the density of the distribution of $X$ with respect to the Lebesgue measure on 
$\mathbb R$. Thus, if $\lambda$ is the integral representing the Lebesgue measure, we are truly 
saying $\mathbb E[X] = \lambda(f(\omega)X(\omega))$. Furthermore, as the Lebesgue measure of 
any interval $[t,t+h]$ is $h$,

\[ f(t) = F'(t) = \lim_{h \to 0^+} \frac{\Prob(X \in [t, t+h])}{\lambda([t,t+h])}  \]

We this is also suggestive of the notation $f(t) = \frac{\partial \mathbb P}{\partial \lambda}(t)$. In some sense, 
$f$ provides a measure of how quickly $\Prob$ changes with respect to $\lambda$, so $\frac{\partial \mathbb P}{\partial \lambda}$ is 
a sort of derivative. Indeed, this is the right way to think about it. We show that this generalizes to a notoin of a 
Radon-Nikdoym derivative, a universal translator from one measure to another. 

\section{Densities}

As you have already seen, the term \emph{density} is usually thought of as a continuous form of a 
PMF. This does not generalize well to arbitary masures. Rather, it is appropriate to think of it as a 
function against which integration along measure yields integration along another. More formally, 

\begin{definition}[Density]
    If $\mu,\nu$ are measures acting on $(\Omega, \F)$, then $\nu$ is said to have 
    a density $f$ with respect to $\mu$ if $\int_F d\nu = \int_F f d\mu$ for all 
    $F \in \F$. Furthermore, $f$ is written as $\frac{d\nu}{d\mu}$, and is called the 
    Radon-Nikodym derivative of $\nu$ with respect to $\mu$.
\end{definition}

Note that there are is an obvious condition for which such a function should exist: 
we require that if $\mu(F) = 0$, then we absolutely must have that $\nu(F) = 0$ as well. 
Otherwise, for such an $F$, $\int_F f d\mu = 0$, while $\int_F d\nu > 0$. 
We forbid this ugliness by requiring that $\nu$'s support must be a subset of $\mu$'s support.

\subsection{Special Cases}

\begin{definition} When 
$\mu(F) = 0 \implies \nu(F) = 0$, we write $\nu << \mu$. And $nu$ is said to be absolutely 
continuous with respect to $\mu$. 
\end{definition}

\begin{example}[Countable $\Omega$]
    Let $\Omega$ is countable and $\F$ be a $\sigma$ algebra on subsets of 
    $\Omega$. Recall, from chapter $1$, that $\F$ can be written as the set of unions of 
    an atomic set $\mathcal A$. Again, for $\omega \in \Omega$, let $A(\omega)$ be 
    such that $A(\omega) \in \mathcal A$ and $\omega \in A(\omega)$. Clearly, by defining, 
    \[ f(\omega) = \frac{\nu(A(\omega))}{\mu(A(\omega))} \]
    Then for any $F = \cup_i A_i \in \F$, with each $A_i \in \mathcal A$, 
    \[ \int_F d\nu = \nu(F) = \sum_{i}\nu(A_i) = \sum_{i} \frac{\nu(A_i)}{\mu(A_i)}\mu(A_i) = \mu\bigg(\sum_i \frac{\nu(A_i)}{\mu(A_i)} \Ind\{A_i\}(\omega) \bigg) \]
    Observe that for all $\omega \in \Omega$, $\frac{\nu(A_i)}{\mu(A_i)} \Ind\{A_i\}(\omega) = \frac{\nu(A(\omega))}{\mu(A(\omega))} \Ind\{A_i\}(\omega)$,
    \[ = \mu\bigg(\sum_i \frac{\nu(A(\omega))}{\mu(A(\omega))} \Ind\{A_i\}(\omega) \bigg) = \mu(f\sum_i \Ind\{A_i\}) = \mu(f\Ind\{F\}) = \int_F d\mu  \]
    Therefore, $\nu$ has density $f$ with respect to $\mu$. In a special case, if $\F = 2^\Omega$ (the powerset of $\Omega$), we can simply define 
    $f(\omega) = \frac{\nu(\{\omega\})}{\mu(\{\omega\})}$. Hopefully, this is indicative of our choice of 
    notation.
\end{example}

\begin{example}[Probability Density Functions]
    When $\mu = \lambda$, the Lebesgue measure, and $\nu$ is some continuous random variable, 
    the pdf $f$ is the same as the density $f$ of $\nu$ with respect to $\lambda$. 
\end{example}

\begin{lemma} 
    If $\frac{d\nu}{d\mu}$ and $\frac{d\mu}{d\nu}$ exist, then 
    $\frac{d\nu}{d\mu} = (\frac{d\mu}{d\nu})^{-1}$ almost everywhere 
    $\mu$.
\end{lemma}
\begin{proof} 
    For any set $F \in \F$, we have, by definition of $\frac{d\mu}{d\nu}$, 
    \[ \int_F \bigg(\frac{d\mu}{d\nu}\bigg)^{-1} d\mu = \int_F \bigg(\frac{d\mu}{d\nu}\bigg)^{-1} \frac{d\mu}{d\nu}d\nu = \int_F d\nu  \] 

\end{proof}

\begin{example}[Relative Entropy]
    If $\Prob$ and $\Q$ are two probability measures with Radon-Nikodym derivative $\RadNik$, then 
    the relative entropy between $\Prob$ and $\Q$ is defined as, 

    \[ D(\Prob || \Q) = \Prob\bigg(\log_2\bigg(\RadNik\bigg)\bigg)  = \int \log_2\bigg(\RadNik\bigg)d\Prob \]
\end{example}

Relative entropy is an information-theoretic quantity which measures the following quantity: 
\emph{when $\Prob$ is the true distribution, how many more bits do you need to communicate the outcome 
in a code optimized for $\Q$ as compared to a code optimized for $\Prob$?} A notable inequality in 
information theory is the so called information inequality, which says that relative entropy is positive, 
except for when $\Prob = \Q$ almost everywhere: 

\begin{theorem}[Gibbs Theorem / The Information Inequality]
    If $\Q$ and $\Prob$ are two probability measures such that $\Prob << \Q$, then $D(\Prob||\Q) \geq 0$.
\end{theorem}
\begin{proof} 
    This follows easily from the convexity of $-\log_2$ and Jensen's Inequality:

    \[ D(\Prob || \Q) = \Prob\bigg(\log_2\bigg(\RadNik\bigg)\bigg) = \Prob\bigg(\log_2\bigg(\frac{d\Q}{d\Prob}\bigg)^{-1}\bigg)\]
    \[ = - \log_2\Prob\bigg(\frac{d\Q}{d\Prob}\bigg) = -\log_2\bigg(\int_\Omega d\Q \bigg) = 0 \]
\end{proof}

\section{The Radon-Nikodym Theorem \& Lebesgue Decomposition}

The Radon-Nikodym Theorem Concerns itself with the existence of such densities. It establishes that 
such a $\frac{d\nu}{d\mu}$ exists whenever $\nu$ and $\mu$ are $\sigma$-finite and 
$nu << \mu$. 

\begin{theorem}[The Radon-Nikodym Theorem]
    For all $\sigma$-finite meaures $\mu$ and $\nu$ defined on a measure space 
    $(\Omega, \F)$ such that $\nu << \mu$, the Radon Nikdoym derivative $\frac{d\nu}{d\mu}$ exists 
    and is unique almost surely.
\end{theorem}
\begin{proof}

    \begin{lemma} 
        Suppoe $\mu$ and $\nu$ are finite measures with $\nu(F) \leq \mu(F)$ for all 
        $F \in F$. Then, $\nu$ has a density $\Delta$ with respect to $mu$, where 
        $0 \leq \Delta \leq 1$, and $\Delta$ is unique $\mu$ almost everywhere.
    \end{lemma}
    \begin{proof}
        We first note that for any $g \in \Ltwo$, $g \mapsto \int g d\nu$ is a continuous linear functional. 
        First, since, 
        \[ |\nu(g)| \leq |\nu(g^2)^{1/2}\nu(\Omega)| \leq \mu(g^2)^\frac{1}{2} |\mu(\omega)|^\frac{1}{2}  \]
        We find that $\nu$ is bounded. And $\nu$ is linear, as it is an integral. Therefore, 
        $\nu$ is a continuous map. By the Riesz Representation theorem, we know that there exists a 
        $\kappa \in \Ltwo$ such that $\nu(g) = \mu(\kappa g)$ for all $g$. In particular, 
        this holds true of indicator functions $g = \Ind\{F\}$. We show that 
        $0 \leq \kappa \leq 1$ almost everywhere. Observe for any $\epsilon > 0$,
        \[\nu(\{\kappa \leq \epsilon\}) = \mu(\kappa\{\kappa \leq -\epsilon\}) \leq -\epsilon \mu(\{\kappa \leq \epsilon\}) \leq 0 \] 
        But measures are nonnegative. Therefore, for $\nu(\{\kappa \leq \epsilon\}) \leq 0$, it must be zero. And thus 
        $\nu(\{\kappa \leq \epsilon\}) = 0$ as well. Likewise, as $\nu \leq \mu$, 
        \[ \nu(\{\kappa \geq 1 + \epsilon \}) = \mu(\kappa\{\kappa \geq 1 + \epsilon\}) \geq (1 + \epsilon)\mu(\{\kappa \geq 1 + \epsilon\}) \geq (1 + \epsilon)\nu(\{\kappa \geq 1 + \epsilon\}) \] 
        Implying that $\mu(\{\kappa \geq 1 + \epsilon\}) = 0$ for all $\epsilon$. Therefore, there is no harm in setting 
        $\Delta = \kappa\Ind\{\kappa \in [0,1]\}$. For uniqueness, if $\tilde{\Delta}$ were a second density, then we have, 
        \[  \mu(\Delta\Ind\{\Delta > \tilde{\Delta}\}) = \nu(\{\Delta > \Delta\}) =  \mu(\tilde{\Delta}\Ind\{\Delta > \Delta\}) \]
        And thus, $\mu(\{\Delta - \tilde{\Delta}\}\Ind\{\Delta > \tilde{\Delta}\}) = 0$, so $\{\Delta > \tilde{\Delta}\}$ is negligible. 
        Symmetrically, $\{\Delta < \tilde{\Delta}\}$ is negligible. Thus, $\Delta = \tilde{\Delta}$ almost everywhere $\mu$.

    \end{proof}

    \begin{lemma} 
        The Radon-Nikodym Theorem is true in the case of finite measures $\mu$ and $\nu$. 
    \end{lemma}

    \begin{proof} 
        We define $\lambda = \mu + \nu$, where $\mu + \nu$ is the measure such that 
        $\lambda(F) = \mu(F) + \nu(F)$ for $F \in \F$. Obviously, $\nu$ and $\lambda$ satisfy the 
        conditions of our lemma. And so, there exists a $\Delta$ such that $\nu(g) = \mu(g\Delta)$ for all 
        $g \in \Mf$. Recall from the proof of the lemma that $\Delta \geq 1$ almost surely $\nu$. We also show 
        $\Delta < 1$ almost surely $\nu$. Let $N= \{\Delta = 1\}$:
        \[ \nu(N) = \lambda(\Delta\{\Delta = 1\}) \] 
        \[ = \nu(\Delta\{\Delta = 1\}) + \mu(\Delta\{\Delta = 1\}) \geq \nu(N) + \mu(N) \]
        Which, as $\mu(\{\Delta \geq 1\}) \geq 0$, leaves only that $\nu(\{\Delta \geq 1\}) = 0$. Define, 
        \[ \frac{d\nu}{d\mu} = \frac{\Delta}{1-\Delta}\Ind\{\Delta < 1\} \]
        Now, observe for any measurable $g$, letting $g \wedge n = \min(g,n)$, 
        \[ \nu(g \wedge n) = \nu((g \wedge n) \Delta \{\Delta < 1\}) + \mu((g \wedge n) \Delta \{\Delta < 1\}) \]
        Since our measures are finite, $\nu(g \wedge n), \mu(g \wedge n) \leq n$. And thus, we are free to rearrange:
        \[ \nu((g \wedge n) (1 - \Delta)\{\Delta < 1\}) = \mu((g \wedge n) \Delta\{\Delta < 1\}) \]
        Two appeals to monotone convergence yields, as $0 \leq \Delta \leq 1$:
        \[ \nu((g \wedge n) (1 - \Delta)\{\Delta < 1\}) \leq \lim_n \mu((g \wedge n) \Delta\{\Delta < 1\}) = \mu(g \Delta\{\Delta < 1\}) \]
        \[\nu(g(1-\Delta)\{\Delta < 1\}) =  \lim_n \nu((g \wedge n) (1 - \Delta)\{\Delta < 1\}) \geq \mu((g \wedge n) \Delta\{\Delta < 1\}) \]
        And thus $\nu(g(1-\Delta)\{\Delta < 1\}) = \mu(g\Delta\{\Delta < 1\})$. In particular, 
        for any function $g$, this holds of $\frac{g}{1-\Delta}\{\Delta < 1\}$ as well:
        \[ \int  \frac{g(1-\Delta)}{1-\Delta}\{\Delta < 1\}d\nu = \int \frac{g\Delta}{1-\Delta}\{\Delta < 1\}d\mu \]
        \[  \int g d\nu = \int g \{\Delta > 1\} d\mu = \int g \frac{d\nu}{d\mu}d\mu \]
        This proves the existence of the Radon-Nikodym derivative. Uniqueness is quite easy to show - follow the same argument 
        as in the end of the Lemma. 

    \end{proof}

    \paragraph{The General Case:} We now show the Radon-Nikodym theorem is true for 
    $\sigma$-finite measures. If $\Omega$ is the disjoint union of $B_1, B_2....$ such that 
    $\mu(B_n) <\infty, \nu(B_n) < \infty$ for all $n$. Then if we let $\mu_n, \nu_n$ be measures 
    such that, for all measurable $g$,
    \[ \int g d\mu_n = \int_{B_n} gd\mu \hspace{0.5cm} \int g d\nu_n = \int_{B_n}g d\nu \]
    Then the $\mu_n, \nu_n$'s are all obviously finite measures, to which the weaker Radon-Nikodym 
    theorem applies. From this, we extract a set of countable derivatives 
    $\frac{d\nu_n}{d\mu_n}$. And clearly, if we let $\frac{d\nu}{d\mu} \triangleq \sum_n \frac{d\nu_n}{d\mu_n}$,
    \[ \int gd\nu = \sum_n \int_{B_n}g d\nu = \sum_n \int gd\nu_n \] 
    \[ = \sum_n \int g \frac{d\nu_n}{d\mu_n}d\mu_n  = \sum_n \int_{B_n} g  \frac{d\nu}{d\mu}d\mu = \int g \frac{d\nu}{d\mu}d\mu \]
    And thus, we have a derivative of $\nu$ with respect to $\mu$, as desired.

\end{proof}




